\PassOptionsToPackage{unicode=true}{hyperref} % options for packages loaded elsewhere
\PassOptionsToPackage{hyphens}{url}
%
\documentclass[]{article}
\usepackage{lmodern}
\usepackage{amssymb,amsmath}
\usepackage{ifxetex,ifluatex}
\usepackage{fixltx2e} % provides \textsubscript
\ifnum 0\ifxetex 1\fi\ifluatex 1\fi=0 % if pdftex
  \usepackage[T1]{fontenc}
  \usepackage[utf8]{inputenc}
  \usepackage{textcomp} % provides euro and other symbols
\else % if luatex or xelatex
  \usepackage{unicode-math}
  \defaultfontfeatures{Ligatures=TeX,Scale=MatchLowercase}
\fi
% use upquote if available, for straight quotes in verbatim environments
\IfFileExists{upquote.sty}{\usepackage{upquote}}{}
% use microtype if available
\IfFileExists{microtype.sty}{%
\usepackage[]{microtype}
\UseMicrotypeSet[protrusion]{basicmath} % disable protrusion for tt fonts
}{}
\IfFileExists{parskip.sty}{%
\usepackage{parskip}
}{% else
\setlength{\parindent}{0pt}
\setlength{\parskip}{6pt plus 2pt minus 1pt}
}
\usepackage{hyperref}
\hypersetup{
            pdftitle={The Woods Hole Assessment Model (WHAM): a general state-space assessment framework that incorporates time- and age-varying processes via random effects and links to environmental covariates},
            pdfauthor={Brian C. Stock1, Timothy J. Miller1},
            pdfborder={0 0 0},
            breaklinks=true}
\urlstyle{same}  % don't use monospace font for urls
\usepackage[margin=1in]{geometry}
\usepackage{graphicx,grffile}
\makeatletter
\def\maxwidth{\ifdim\Gin@nat@width>\linewidth\linewidth\else\Gin@nat@width\fi}
\def\maxheight{\ifdim\Gin@nat@height>\textheight\textheight\else\Gin@nat@height\fi}
\makeatother
% Scale images if necessary, so that they will not overflow the page
% margins by default, and it is still possible to overwrite the defaults
% using explicit options in \includegraphics[width, height, ...]{}
\setkeys{Gin}{width=\maxwidth,height=\maxheight,keepaspectratio}
\setlength{\emergencystretch}{3em}  % prevent overfull lines
\providecommand{\tightlist}{%
  \setlength{\itemsep}{0pt}\setlength{\parskip}{0pt}}
\setcounter{secnumdepth}{5}
% Redefines (sub)paragraphs to behave more like sections
\ifx\paragraph\undefined\else
\let\oldparagraph\paragraph
\renewcommand{\paragraph}[1]{\oldparagraph{#1}\mbox{}}
\fi
\ifx\subparagraph\undefined\else
\let\oldsubparagraph\subparagraph
\renewcommand{\subparagraph}[1]{\oldsubparagraph{#1}\mbox{}}
\fi

% set default figure placement to htbp
\makeatletter
\def\fps@figure{htbp}
\makeatother

\usepackage{url}
\usepackage{setspace}
%\singlespacing
%\onehalfspacing
\doublespacing
\usepackage{lineno}
\linenumbers
\usepackage[belowskip=0pt,aboveskip=0pt]{caption}
\usepackage{booktabs}
\usepackage{longtable}
\usepackage{array}
\usepackage{multirow}
\usepackage{wrapfig}
\usepackage{float}
\usepackage{colortbl}
\usepackage{pdflscape}
\usepackage{tabu}
\usepackage{threeparttable}
\usepackage{threeparttablex}
\usepackage[normalem]{ulem}
\usepackage{makecell}
\usepackage{xcolor}

\title{The Woods Hole Assessment Model (WHAM): a general state-space assessment
framework that incorporates time- and age-varying processes via random
effects and links to environmental covariates}
\author{Brian C. Stock\textsuperscript{1}, Timothy J. Miller\textsuperscript{1}}
\date{}

\begin{document}
\maketitle

\(^1\)\href{mailto:brian.stock@noaa.gov}{\nolinkurl{brian.stock@noaa.gov}},
\href{mailto:timothy.j.miller@noaa.gov}{\nolinkurl{timothy.j.miller@noaa.gov}},
Northeast Fisheries Science Center, National Marine Fisheries Service,
166 Water Street, Woods Hole, MA 02543, USA\\

\pagebreak

\hypertarget{abstract}{%
\subsection*{Abstract}\label{abstract}}
\addcontentsline{toc}{subsection}{Abstract}

The rapid changes observed in many marine ecosystems that support
fisheries pose a challenge to stock assessment and management predicated
on time-invariant productivity and considering species in isolation. In
single-species assessments, two main approaches have been used to
account for productivity changes: allowing biological parameters to vary
stochastically over time (empirical), or explicitly linking population
processes such as recruitment (\emph{R}) or natural mortality (\emph{M})
to environmental covariates (mechanistic). Here, we describe the Woods
Hole Assessment Model (WHAM) framework and software package, which
combines these two approaches. WHAM can estimate time- and age-varying
random effects on annual transitions in numbers at age (NAA), \emph{M},
and selectivity, as well as fit environmental time-series with process
and observation errors, missing data, and nonlinear links to \emph{R}
and \emph{M}. WHAM can also be configured as a traditional statistical
catch-at-age (SCAA) model in order to easily bridge from status quo
models and test them against models with state-space and environmental
effects, all within a single framework.

We fit models with and without (independent or autocorrelated) random
effects on NAA, \emph{M}, and selectivity to data from five stocks with
a broad range of life history, fishing pressure, number of ages, and
time-series length. Models that included random effects performed well
across stocks and processes, especially random effects models with a two
dimensional (2D) first-order autoregressive (AR1) covariance structure
over age and year. We conducted simulation tests and found negligible or
no bias in estimation of important assessment outputs (SSB, \emph{F},
stock status, and catch) when the operating and estimation models
matched. However, bias in SSB and \emph{F} was often non-trivial when
the estimation model was less complex than the operating model,
especially when models without random effects were fit to data simulated
from models with random effects. Bias of the variance and correlation
parameters controlling random effects was also negligible or slightly
negative as expected. Our results suggest that WHAM can be a useful tool
for stock assessment when environmental effects on \emph{R} or \emph{M},
or stochastic variation in NAA transitions, \emph{M}, or selectivity are
of interest. In the U.S. Northeast, where the productivity of several
groundfish stocks has declined, conducting assessments in WHAM with
time-varying processes via random effects or environment-productivity
links may account for these trends and potentially reduce retrospective
bias.

\hypertarget{keywords}{%
\subsubsection*{Keywords}\label{keywords}}
\addcontentsline{toc}{subsubsection}{Keywords}

state-space; stock assessment; random effects; time-varying;
environmental effects; recruitment; natural mortality; Template Model
Builder (TMB)

\pagebreak

\hypertarget{introduction}{%
\section{Introduction}\label{introduction}}

The last two decades have increasingly seen a push for more holistic,
ecosystem-based fisheries management (Larkin 1996; Link 2002). In part,
this is a recognition that considering single species in isolation
produces riskier and less robust outcomes long-term (Patrick and Link
2015). In several high-profile cases, fisheries management has failed to
prevent collapses because they did not reduce fishing pressure in
responses to changes in natural mortality (\(M\)), recruitment, or
migration patterns caused by dynamics external to the stock in question
(Northern cod: Shelton et al. 2006; Rose and Rowe 2015; Gulf of Maine
cod: Pershing et al. 2015; Pacific sardine: Zwolinski and Demer 2012).
This is particularly concerning in the context of climate change and the
wide range of biological processes---often assumed to be constant---in
stock assessments that are likely to be affected (Stock et al. 2011;
Tommasi et al. 2017).

One approach to account for changing productivity is to explicitly link
population processes to environmental covariates in single-species stock
assessments, i.e.~the mechanistic approach \emph{sensu} Punt et al.
(2014). Traditional single-species assessments are based on internal
population dynamics and the effect of fishing mortality (\(F\)), even
though fisheries scientists have long known about important drivers of
time-varying population processes, e.g.~recruitment, mortality, growth,
and movement (Garstang 1900; Hjort 1914). Effects of the environment or
interactions with other species can be considered contextually, rather
than explicitly as estimated parameters (although temporal variation in
empirical weight and maturity at age can affect reference points).
Despite how counterintuitive this may seem to ecologists and
oceanographers who study such relationships, the evidence for direct
linkages to specific environmental covariates is often weak and can
break down over time (Myers 1998; McClatchie et al. 2010). Additionally,
the primary goal of most assessments is to provide management advice on
near-term sustainable harvest levels---not to explain ecological
relationships. Even if an environmental covariate directly affects fish
productivity, including the effect in an assessment may not improve
management advice if the effect is weak (De Oliveira and Butterworth
2005). Worse, including environmental effects in an assessment or
management system has been shown to actually provide worse management in
some cases (Walters and Collie 1988; De Oliveira and Butterworth 2005;
Punt et al. 2014). This can be true even in cases of relatively
well-understood mechanistic links between oceanic conditions and fish
populations, as in the case of sea surface temperature and Pacific
sardine (Zwolinski and Demer 2012; Hill et al. 2018). Still,
incorporating mechanistic environment-productivity links in assessments
does have the potential to reduce residual variance, particularly in
periods when few demographic data exist (Shotwell et al. 2014; Miller et
al. 2016).

An alternative approach is to allow biological parameters to vary
stochastically over time, i.e.~the empirical approach \emph{sensu} Punt
et al. (2014). In this case, the variation is caused by a range of
sources that are not explicitly modeled. Statistical catch-at-age (SCAA)
models typically only estimate year-specific recruitment (\(R_t\)) and
\(F_t\), often as deviations from a mean, \(R_0\), that may or may not
be a function of spawning biomass, e.g.
\(\text{log}R_t = \text{log}R_0 + \epsilon_t\). The main reason that
other parameters are assumed constant is simply that there are not
enough degrees of freedom to estimate many time-varying parameters. A
common solution is to penalize the deviations, e.g.
\(\epsilon_t \sim \mathcal{N}(0,\sigma^2_\epsilon)\), although the
penalty terms, \(\sigma^2_\epsilon\), must be fixed or iteratively tuned
and are therefore somewhat subjective (Methot and Taylor 2011; Methot
and Wetzel 2013; Aeberhard et al. 2018; Xu et al. 2020). State-space
models that treat parameters as unobserved states can, in principle,
avoid such subjectivity by estimating the penalty terms as variance
parameters constraining random effects and maximizing the marginal
likelihood (Thorson 2019). In this way, state-space models can allow
processes to vary in time while simultaneously estimating fewer
parameters.

Although state-space stock assessments have existed for some time
(Mendelssohn 1988; Sullivan 1992; Gudmundsson 1994), the recent
development of Template Model Builder (TMB, Kristensen et al. 2016)
software to perform efficient Laplace approximation has greatly expanded
their use (Nielsen and Berg 2014; Cadigan 2016; Miller et al. 2016). In
addition to the key advantage of objectively estimating variance, or
``data weighting'', parameters, state-space models naturally predict
unobserved states, and therefore handle missing data and short-term
projections in a straightforward way (ICES 2020). In comparisons with
SCAA models, state-space models have been shown to have larger, more
realistic, uncertainty and lower retrospective bias (Miller and Hyun
2018; Stock et al. n.d.).

Retrospective bias can occur when changing environmental conditions lead
to changes in productivity that are unaccounted for in stock
assessments, and this is a concern common to several groundfish stocks
on the Northeast U.S. Shelf (Brooks and Legault 2016; Tableau et al.
2018). The Northeast U.S. Shelf ecosystem is rapidly changing, and this
has motivated managers to make the ``continue{[}d{]} development of
stock assessment models that include environmental terms'' a top
priority (Hare et al. 2016). Applications of state-space models with
environmental effects on recruitment, growth, \(M\), and maturity have
proven promising (Xu et al. 2018; Miller et al. 2018; Miller and Hyun
2018; O'Leary et al. 2019). In addition to providing short-term (1-3
years) catch advice with reduced retrospective bias, it is hoped that
environment-linked assessments will help create realistic evaluation of
sustainable stock and harvest levels in the medium-term (3-10 years) for
stocks that have not rebounded in response to dramatic decreases in
\(F\).

To address the needs of fisheries management in a changing climate, we
seek an assessment framework that combines both the empirical and
mechanistic approaches. Namely, it should be able to 1) estimate
time-varying parameters as random effects (i.e.~a state-space model),
and 2) include environmental effects directly on biological parameters.
The framework should also allow for easy testing against status quo SCAA
models to ease gradual adoption through the ``research track'' or
``benchmark'' assessment process (Lynch et al. 2018). The objectives of
this manuscript are to introduce the Woods Hole Assessment Model (WHAM)
framework and demonstrate its ability to:

\begin{enumerate}
\def\labelenumi{\arabic{enumi}.}
\tightlist
\item
  estimate time- and age-varying random effects on annual changes in
  abundance at age, \(M\), and selectivity;
\item
  fit environmental time-series with process and observation error,
  missing data, and a link to a population process; and
\item
  simulate new data and random effects to conduct self- and cross-tests
  (\emph{sensu} Deroba et al. 2015) to estimate bias in parameters and
  derived quantities.
\end{enumerate}

Throughout, we describe how the above are implemented using the
open-source WHAM software package (Miller and Stock 2020).

\hypertarget{methods}{%
\section{Methods}\label{methods}}

\hypertarget{model-description}{%
\subsection{Model description}\label{model-description}}

WHAM is a generalization and extension of Miller et al. (2016) in TMB.
It is in many respects similar to the Age-Structured Assessment Program
(ASAP, Legault and Restrepo 1998; Miller and Legault 2015) and can be
configured to fit statistical catch-at-age models nearly identically.
There is functionality built into WHAM to migrate ASAP input files to R
inputs needed for WHAM, and WHAM uses many of the same types of data
inputs, such as empirical weight-at-age, so that existing assessments in
the U.S. Northeast can be easily replicated and tested against models
with state-space and environmental effects in a single framework.

\hypertarget{processes-with-random-effects}{%
\subsubsection{Processes with random
effects}\label{processes-with-random-effects}}

WHAM primarily diverges from ASAP through the implementation of random
effects on three processes: inter-annual transitions in numbers at age
(\emph{NAA}), natural mortality (\emph{M}), and selectivity (\emph{s}),
as well as allowing effects of environmental covariates (\emph{Ecov}) on
recruitment and natural mortality (but see ASAP4; Miller and Legault
2015). The environmental covariates and their observations are treated
using state-space models with true, unobserved values treated as random
effects and observation on them having error. Other than environmental
covariates, the processes are assumed to have a two dimensional (2D)
first-order autoregressive (AR1) covariance structure over age and year,
although correlation in either or both dimensions can be turned off. The
2D AR1 structure has been widely used to model deviations by age and
year in the parameters \(F_{a,y}\) (Nielsen and Berg 2014), \(M_{a,y}\)
(Cadigan 2016; Stock et al. n.d.), \(s_{a,y}\) (Xu et al. 2019), and
\(N_{a,y}\) (Stock et al. n.d.), as well as in the catch (\(C_{a,y}\))
and survey index (\(I_{a,y}\)) observations (Berg and Nielsen 2016).

\hypertarget{numbers-at-age-naa}{%
\paragraph{\texorpdfstring{Numbers at age
(\emph{NAA})}{Numbers at age (NAA)}}\label{numbers-at-age-naa}}

The stock equations in WHAM that describe the transitions between
numbers at age are identical to Miller et al. (2016) and Nielsen and
Berg (2014):

\begin{equation}
\label{eq:NAA}
  \text{log}N_{a,y}=\left\{
    \begin{array}{@{}lll@{}}
      \text{log} \left( f(SSB_{y-1}) \right) + \varepsilon_{1,y}, & \text{if}\ a = 1 \\
      \text{log} \left( N_{a-1,y-1} \right) - Z_{a-1,y-1} + \varepsilon_{a,y}, & \text{if}\ 1 < a < A \\
      \text{log} \left( N_{A-1,y-1} e^{-Z_{A-1,y-1}} + N_{A,y-1} e^{-Z_{A,y-1}} \right) + \varepsilon_{A,y}, & \text{if}\ a = A
    \end{array}\right.
\end{equation}

where \(N_{a,y}\) are the numbers at age \emph{a} in year \emph{y},
\emph{Z} is the total mortality rate (\(F + M\)), \(f\) is the
stock-recruit function, \emph{Y} is the total number of observation and
prediction years, and \emph{A} represents the plus-group. In this
analysis we demonstrate four possible models for the NAA deviations,
\(\varepsilon_{a,y}\).

m1 is most similar to a SCAA model, where only recruitment deviations,
\(\varepsilon_{1,y}\), are estimated (i.e. \(\varepsilon_{a,y} = 0\) for
\(a > 1\) in Eqn \ref{eq:NAA}). In m1, the recruitment deviations are
assumed to be independent and identically distributed (IID):

\[\varepsilon_{1,y} \sim \mathcal{N}\left( - \frac{\sigma^2_R}{2}, \sigma^2_R \right)\]
The \(- \frac{\sigma^2_R}{2}\) bias correction term is included by
default so that the expected recruitment, \(E(N_{1,y} = R_y)\) equals
the expected \(R_y\) from the stock-recruit function (Methot and Taylor
2011; Thorson 2019). This bias correction adjustment can also be turned
off. The only difference between m1 and a SCAA is that annual
recruitments are random effects and \(\sigma^2_R\) is an estimated
parameter within the model.

m2 is the same as m1, except that the recruitment deviations are
stationary AR1 with autocorrelation parameter \(-1<\rho_y<1\):

\[\varepsilon_{1,y+1} \sim \mathcal{N}\left(\rho_y \varepsilon_{1,y} - \frac{\sigma^2_R}{2 (1-\rho^2_y)}, \sigma^2_R \right)\]

m3 is the ``full state-space'' model from Nielsen and Berg (2014) and
Miller et al. (2016), where all numbers at age are independent random
effects and:

\begin{equation}
  \varepsilon_{a,y} \sim \left\{
    \begin{array}{@{}ll@{}}
      \mathcal{N} \left( - \frac{\sigma^2_R}{2}, \sigma^2_R \right), & \text{if}\ a = 1 \\
      \mathcal{N} \left( - \frac{\sigma^2_a}{2}, \sigma^2_a \right), & \text{if}\ a > 1
    \end{array}\right.
\end{equation}

where \(\sigma^2_a\) for all ages \(a > 1\) are assumed to be the same
but different from age \(a = 1\), i.e.~recruitment. This assumption is
sensible because variability of deviations between expected and realized
recruitment are typically larger than deviations from expected abundance
at older ages.

m4 treats the numbers at all ages as random effects, as in m3, but the
NAA deviations, \(\varepsilon_{a,y}\), have a 2D stationary AR1
structure as in Stock et al. (n.d.):

\[\mathbf{E} \sim \mathcal{MVN} \left( 0, \Sigma \right)\]

where
\(\mathbf{E} = (\varepsilon_{1,1}, \ldots, \varepsilon_{1,Y-1}, \varepsilon_{2,1}, \ldots, \varepsilon_{2,Y-1}, \ldots, \varepsilon_{A,1}, \ldots, \varepsilon_{A,Y-1})'\)
is a vector of all NAA deviations, \(\boldsymbol{\Sigma}\) is the
covariance matrix of \(\mathbf{E}\) defined by:

\[ \text{Cov} \left( \varepsilon_{a,y}, \varepsilon_{\tilde{a},\tilde{y}} \right) = \frac{\sigma_a \sigma_{\tilde{a}} \rho^{|a-\tilde{a}|}_{a} \rho^{|y-\tilde{y}|}_{y}}{\left(1-\rho^2_{a}\right) \left(1-\rho^2_{y}\right)}\]
and \(-1<\rho_a<1\) and \(-1<\rho_y<1\) are the AR1 coefficients in age
and year, respectively. As in m3, \(\sigma^2_a\) for all ages \(a > 1\)
are assumed to be the same but different from age \(a = 1\),
\(\sigma^2_R\). The bias correction term for age \(a > 1\) in m4 is
\(- \frac{\sigma^2_a}{2 (1-\rho^2_y)(1-\rho^2_a)}\).

\hypertarget{natural-mortality-m}{%
\paragraph{\texorpdfstring{Natural mortality
(\emph{M})}{Natural mortality (M)}}\label{natural-mortality-m}}

For natural mortality, there are mean parameters for each age,
\(\mu_{M_a}\), each of which may be estimated freely or fixed at the
initial values. The \(\mu_{M_a}\) may also be estimated in sets of ages,
e.g.~estimate one mean \emph{M} shared across ages 3-5,
\(\mu_{M_3} = \mu_{M_4} = \mu_{M_5}\). There is also an option for
\emph{M} to be specified as a function of weight-at-age,
\(M_{a,y} = \mu_M \text{W}^b_{a,y}\), as in Lorenzen (1996) and Miller
and Hyun (2018). Regardless of whether \(\mu_{M_a}\) are fixed or
estimated, WHAM can also be configured to estimate deviations in
\emph{M}, \(\delta_{a,y}\), as random effects analogous to the NAA
deviations (Cadigan 2016; Stock et al. n.d.):

\begin{equation}
  \begin{array}{cc}
    \text{log}\left( M_{a,y} \right) = \mu_{M_a} + \delta_{a,y} \\
    \text{Cov} \left( \delta_{a,y}, \delta_{\tilde{a},\tilde{y}} \right) = \frac{\sigma^2_M \varphi^{|a-\tilde{a}|}_{a} \varphi^{|y-\tilde{y}|}_{y}}{\left(1-\varphi^2_{a}\right) \left(1-\varphi^2_{y}\right)}
  \end{array}
\end{equation}

where \(\sigma^2_M\), \(\varphi_a\), and \(\varphi_y\) are the AR1
variance and correlation coefficients in age and year, respectively.

In this analysis, we demonstrate three alternative \emph{M} random
effects models. For simplicity, all models treat \(\mu_{M_a}\) as known,
as in most of the original assessments. m1 is identical to the base
\emph{NAA} model, with no random effects on \emph{M}
(\(\sigma^2_M = \varphi_a = \varphi_y = 0\) and not estimated). m2
allows IID \emph{M} deviations, estimating \(\sigma^2_M\) but fixing
\(\varphi_a = \varphi_y = 0\). m3 estimates the full 2D AR1 structure
for \emph{M} deviations.

\hypertarget{selectivity-s}{%
\paragraph{\texorpdfstring{Selectivity
(\emph{s})}{Selectivity (s)}}\label{selectivity-s}}

As in ASAP and many other SCAA assessment frameworks, WHAM assumes
separability in the fishing mortality rate by age and year, e.g.
\(F_{a,y} = F_y s_a\), where \(F_y\) is the ``fully selected'' fishing
mortality rate in year \emph{y} and \(s_a\) is the selectivity at age
\emph{a}. We note that this differs from the approach in SAM (Nielsen
and Berg 2014), where the \(F_{a,y}\) are estimated directly as
multivariate random effects without the separability assumption. Three
parametric forms are available (logistic, double-logistic, and
decreasing-logistic), as well as a non-parametric option to estimate
each \(s_a\) individually (``age-specific''). To allow for temporal
changes in selectivity as in ASAP, WHAM can estimate selectivity in
user-specified time blocks. WHAM estimates selectivity parameters on the
logit scale to avoid boundary problems during estimation.

WHAM estimates annual full \(F_y\) and mean selectivity parameters as
fixed effects. Deviations in selectivity parameters can be estimated as
random effects, \(\zeta_{p,y}\), with autocorrelation by parameter
(\emph{p}), year (\emph{y}), both, or neither. This is done similarly to
Xu et al. (2019), except that the deviations are placed on the
parameters instead of the mean \(s_{a,y}\) in order to guarantee that
\(0 < s_{a,y} < 1\). For example, logistic selectivity with two
parameters \(a_{50}\) and \(k\) is estimated as:

\begin{equation}
  \begin{array}{cccc}
    s_{a,y} = \frac{1}{1 + e^{-(a - a_{{50}_y}) / k_y}} \\
    a_{{50}_y} = l_{a_{50}} + \frac{u_{a_{50}} - l_{a_{50}}}{1 + e^{-(\nu_1 + \zeta_{1,y})}} \\
    k_y = l_k + \frac{u_k - l_k}{1 + e^{-(\nu_2 + \zeta_{2,y})}} \\
    \text{Cov} \left( \zeta_{1,y}, \zeta_{2,\tilde{y}} \right) = \frac{\sigma^2_s \phi_p \phi^{|y-\tilde{y}|}_{y}}{\left(1-\phi^2_{p}\right) \left(1-\phi^2_{y}\right)}
  \end{array}
\end{equation}

where \(\nu_1\) is the logit-scale mean \(a_{50}\) parameter with lower
and upper bounds \(l_{a_{50}}\) and \(u_{a_{50}}\), \(\nu_2\) is the
logit-scale mean \(k\) parameter with lower and upper bounds \(l_k\) and
\(u_k\), \(\sigma^2_s\) is the AR1 variance, and \(\phi_p\), and
\(\phi_y\) are the AR1 correlation coefficients by parameter and year.

Below, we demonstrate three models with random effect deviations on
logistic selectivity, akin to those for \emph{M}. m1 treats all numbers
at age as independent random effects (i.e. \emph{NAA} m3) but with no
random effects on \emph{s} (\(\sigma^2_s = \phi_p = \phi_y = 0\) and not
estimated). m2 allows IID \emph{s} deviations, estimating \(\sigma^2_s\)
but fixing \(\phi_p = \phi_y = 0\). m3 estimates the full 2D AR1
structure for \emph{s} deviations.

\hypertarget{environmental-covariates-ecov}{%
\paragraph{\texorpdfstring{Environmental covariates
(\emph{Ecov})}{Environmental covariates (Ecov)}}\label{environmental-covariates-ecov}}

WHAM models environmental covariate data using state-space models with
process and observation components. The true, unobserved values (or
``latent states'', \(X_y\)) are then linked to the population dynamics
equations with user-specified lag. For example, recruitment in year
\emph{y} may be influenced by \(X_{y-1}\) (lag 1), while natural
mortality in year \emph{y} may be influenced by \(X_y\) (lag 0).
Multiple environmental covariates may be included, but only as
independent processes. The \emph{Ecov} and population model years do not
need to match, and missing years are allowed. In particular, including
\emph{Ecov} data in the projection period can be useful.

\hypertarget{process-model}{%
\subparagraph{Process model}\label{process-model}}

There are currently two options in WHAM for the \emph{Ecov} process
model: a normal random walk and AR1. We model the random walk as in
Miller et al. (2016):

\[X_{y+1} | X_y \sim \mathcal{N}\left( X_y, \sigma^2_X\right)\]

where \(\sigma^2_X\) is the process variance and \(X_1\) is estimated as
a fixed effect parameter. One disadvantage of the random walk is that
its variance is nonstationary. In short-term projections, \(\hat{X}_y\)
will be equal to the last estimate with an observation and the
uncertainty of \(\hat{X}_y\) will increase over time. If \(\hat{X}_y\)
influences reference points, this leads to increasing uncertainty in
stock status over time as well (Miller et al. 2016).

For this reason, we generally prefer to model \(X_y\) as a stationary
AR1 process as in Miller et al. (2018):

\begin{equation}
  \begin{array}{cc}
    X_1 \sim \mathcal{N} \left( \mu_X, \frac{\sigma^2_X}{1-\phi^2_X} \right) \\
    X_y \sim \mathcal{N} \left( \mu_X(1-\phi_X) + \phi_X X_{y-1}, \sigma^2_X \right)
  \end{array}
\end{equation}

where \(\mu_X\), \(\sigma^2_X\), and \(|\phi_X| < 1\) are the marginal
mean, variance, and autocorrelation parameters. In addition to having
stationary variance, another important difference between the random
walk and AR1 in short-term projections is that the AR1 will gradually
revert to the mean over time, unless environmental covariate
observations are included in the projection period.

\hypertarget{observation-model}{%
\subparagraph{Observation model}\label{observation-model}}

The environmental covariate observations, \(x_y\), are assumed to be
normally distributed with mean \(X_y\) and variance \(\sigma^2_{x_y}\):

\[x_y | X_y \sim \mathcal{N}\left( X_y, \sigma^2_{x_y} \right)\]

The observation variance in each year, \(\sigma^2_{x_y}\), can be
treated as known with year-specific values (as in Miller et al. 2016) or
one overall value shared among years. They can also be estimated as
parameters, likewise either as yearly values or one overall value. If
yearly \(\sigma^2_{x_y}\) estimates are desired, WHAM estimates the
hyperparameters \(\mu_{\sigma_x}\) and \(\sigma_{\sigma_x}\) as fixed
effects and treats the \(\sigma^2_{x_y}\) as random effects:

\[\sigma^2_{x_y} \sim \mathcal{N} \left( \mu_{\sigma_x}, \sigma^2_{\sigma_x} \right)\]

\hypertarget{link-to-population}{%
\subparagraph{Link to population}\label{link-to-population}}

WHAM currently provides options to link the modeled environmental
covariate, \(X_y\), to the population dynamics via recruitment or
natural mortality. It is also sometimes useful to fit the \emph{Ecov}
model without a link to the population dynamics so that models with and
without environmental effects have the same data in the likelihood and
can be compared via AIC.

In the case of recruitment, the options follow the framework laid out by
Fry (1971) and Iles and Beverton (1998): ``controlling''
(density-independent mortality), ``limiting'' (carrying capacity effect,
e.g. \(X_y\) determines the amount of suitable habitat), ``lethal''
(threshold effect, i.e. \(R_t\) goes to 0 at some \(X_y\) value),
``masking'' (\(X_y\) decreases \(\text{d}R/\text{d}SSB\), as expected if
\(X_y\) affects metabolism or growth), and ``directive''
(e.g.~behavioral). Of these, WHAM currently allows controlling,
limiting, or masking effects in the Beverton-Holt stock-recruit
function, and controlling or masking effects in the Ricker function. For
natural mortality, environmental effects are placed on \(\mu_M\), shared
across ages by default.

Regardless of where the environment-population link is, the effect can
be either linear or polynomial. Nonlinear effects of environmental
covariates are common in ecology, and quadratic effects are to be
expected in cases where intermediate values are optimal (Brett 1971;
Agostini et al. 2008). WHAM includes a function to calculate orthogonal
polynomials in TMB, akin to the \texttt{poly()} function in \texttt{R}.

In this analysis, we compare five models with limiting effects on
Beverton-Holt recruitment:

\[\hat{R}_{y+1} = \frac{\alpha \text{SSB}_{y}}{1 + e^{\beta_0 + \beta_1 X_{y} + \beta_2 X^2_{y}} \text{SSB}_y}\]

where \(\text{SSB}_y\) is spawning stock biomass in year \emph{y},
\(\alpha\) and \(\beta_0\) are the standard parameters of the
Beverton-Holt function, and \(\beta_1\) and \(\beta_2\) are polynomial
effect terms that modify \(\beta_0\) based on the value of the estimated
environmental covariate, \(X_y\).

m1 treats \(X_y\) as a random walk (\(\phi_X = 1\)) but does not include
an effect on recruitment (\(\beta_1 = \beta_2 = 0\)). We include m1 in
order to compare AIC of the original model to those with environmental
effects on recruitment. m2 and m3 also treat \(X_y\) as a random walk,
but m2 estimates \(\beta_1\) and m3 estimates both \(\beta_1\) and
\(\beta_2\). m4 and m5 estimate \(X_y\) as an AR1 process instead of a
random walk (estimate \(\phi_X\)), and m4 estimates \(\beta_1\) and m5
estimates both \(\beta_1\) and \(\beta_2\).

\hypertarget{population-observation-model}{%
\subsubsection{Population observation
model}\label{population-observation-model}}

Like ASAP, there are observation likelihood components for aggregate
catch and abundance index for each fleet and index, and age composition
for each fleet and index.

\hypertarget{aggregate-catch-and-indices}{%
\paragraph{Aggregate catch and
indices}\label{aggregate-catch-and-indices}}

The predicted catch at age for fleet \(i\), \(\hat{C}_{a,y,i}\), is a
function of \(N_{a,y}\), \(M_{a,y}\), \(F_{y,i}\), \(s_{a,y,i}\), and
empirical weight at age, \(W_{a,y,i}\):

\[
\hat{C}_{a,y,i} = N_{a,y} W_{a,y,i}\left(1- e^{-Z_{a,y}}\right)\frac{F_{a,y,i}}{Z_{a,y}}.
\]

The log-aggregate catch \(\hat{C}_{y,i} = \sum_a \hat{C}_{a,y,i}\)
observation is assumed to have a normal distribution \begin{equation}
\label{eq:catch}
  \begin{array}{ccc}
    \text{log}(C_{y,i}) \sim \mathcal{N}\left( \text{log}(\hat{C}_{y,i}) - \frac{\sigma^2_{C_{y,i}}}{2}, \sigma^2_{C_{y,i}}\right).
  \end{array}
\end{equation} where the standard deviation \[
\sigma_{C_{y,i}} = e^{\eta_i}\sigma_{\tilde{C}_{y,i}}
\] is a function of an input standard deviation
\(\sigma_{\tilde{C}_{y,i}}\) and a fleet-specific parameter \(\eta_i\)
that is fixed at 0 by default, but may be estimated. The bias correction
term, \(- \frac{\sigma^2_{C_{y,i}}}{2}\), is included by default based
on Aldrin et al. (2020) but can be turned off.

Observations of aggregate indices of abundance are handled identically
to the aggregate catch as in Eqn. \ref{eq:catch} except that for index
\(i\) the predicted index at age is
\[\hat{I}_{a,y,i} = q_i s_{a,y,i} N_{a,y}W_{a,y,i} e^{-Z_{a,y}f_{y,i}}\]
where \(q_i\) is the catchability and \(f_{y,i}\) is the fraction of the
annual time step elapsed when the index is observed. There are options
for indices to be in terms of abundance (numbers) or biomass and
\(W_{a,y,i} = 1\) for the former.

\hypertarget{catch-and-index-age-composition}{%
\paragraph{Catch and index age
composition}\label{catch-and-index-age-composition}}

WHAM includes several options for the catch and index age compositions
including multinomial (default), Dirichlet, Dirichlet-multinomial,
logistic normal and a zero-one inflated logistic normal. In all of the
applications and simulation studies here, we assumed a logistic normal
distribution for age composition observations.

\hypertarget{projections}{%
\subsubsection{Projections}\label{projections}}

The default settings for short-term projections follow common practice
for stock assessments in the U.S. Northeast: the population is projected
three years using the average selectivity, maturity, weight, and natural
mortality at age from the last five model years to calculate reference
points (NEFSC 2020a). WHAM implements similar options as ASAP for
specifying \(F_y\) in the projection years: terminal year \(F_y\),
average \(F\) over specified years, \(F_{X\%}\) (\(F\) at X\% SPR, where
X is specified and 40 by default), user-specified \(F_y\), or \(F\)
derived from user-specified catch. For all options except user-specified
\(F_y\), the uncertainty in projected F is propagated into the
uncertainty of projected population attributes. For models with random
effects on \emph{NAA}, \emph{M}, or \emph{Ecov}, the default is to
continue the stochastic process into the projection years. WHAM does not
currently do this for selectivity because, like \(F\), it is a function
of management. Instead, selectivity is taken as the average of recent
model years.

If the \emph{Ecov} data extend beyond the population model years, WHAM
will fit the \emph{Ecov} model to all available data and use the
estimated \(X_y\) in projections. This may often be the case because
lags can exist in both the physical-biological mechanism and the
assessment process. As an example, a model with an \emph{Ecov} effect on
recruitment may link physical oceanographic conditions, e.g.~surface or
bottom temperature, in year \(t\) to recruitment in year \(t+1\), and
the assessment conducted in year \(t\) may only use population data
through year \(t-1\). In this case, 3-year population projections only
need the \emph{Ecov} model to be projected one year. While the default
handling of \emph{Ecov} projections is to continue the stochastic
process, WHAM includes options to use terminal year \(x_y\), \(x_y\)
averaged over specified years, or specified \(x_y\). The option to
specify \(x_y\) allows users to investigate how alternative climate
projections may affect the stock.

\hypertarget{fits-to-original-datasets}{%
\subsection{Fits to original datasets}\label{fits-to-original-datasets}}

We fit the models described above to data from five stocks with a broad
range of life history, status, and model dimension (number of ages and
years): Southern New England-Mid Atlantic (SNEMA) yellowtail flounder,
butterfish, North Sea cod, Icelandic herring, and Georges Bank (GB)
haddock (Tables \ref{tab:model-descriptions} and \ref{tab:stock-list}).
We assumed the same form of logistic normal for SNEMA yellowtail
flounder age composition observations as Miller et al. (2016) where
unobserved ages are pooled with adjascent ages, but for other stocks
unobserved ages are treated as missing. The variance parameters
associated with the logistic normal distributions were estimated for all
stocks. We fit the \emph{NAA} random effects models to all five stocks
since these represent core WHAM functionality. We chose to highlight one
stock each for the \emph{M}, \emph{s}, and \emph{Ecov} processes because
all models did not converge for all stocks and processes: butterfish
(\emph{M}), GB haddock (\emph{s}), and SNEMA yellowtail flounder
(\emph{Ecov}). We fixed \(\mu_{M_a}\) at the values used, or estimated
as the case for butterfish, in the original assessments. Except for the
GB haddock \emph{s} models, we used the same selectivity
parameterization and time blocks as in the original assessments. We used
the m3 \emph{NAA} model (all NAA deviations are IID random effects) in
the \emph{s} and \emph{Ecov} demonstrations, but the m1 \emph{NAA} model
(only recruitment deviations are IID random effects) for the \emph{M}
demonstrations, because the NAA transitions can be interpreted as
survival and including random effect deviations on both NAA and \emph{M}
can lead to model non-convergence (Stock et al. n.d.). For the SNEMA
yellowtail flounder \emph{Ecov} models, we used the Cold Pool Index
(CPI) as calculated in Miller et al. (2016). We updated the CPI through
2018, except that the 2017 observation was missing because the NEFSC
fall bottom trawl survey was not completed in the SNEMA region. Within
each process, we compared the performance of the different random
effects models using AIC.

We fit all models using the open-source statistical software R (R Core
Team 2020) and TMB (Kristensen et al. 2016), as implemented in the R
package WHAM (Miller and Stock 2020). Documentation and tutorials for
how to specify additional random effect structures in WHAM are available
at \url{https://timjmiller.github.io/wham/}. Code and data files to run
the analysis presented here are available at
\url{https://github.com/brianstock-NOAA/wham-sim}.

\hypertarget{simulation-tests}{%
\subsection{Simulation tests}\label{simulation-tests}}

After fitting each model to the original datasets, we used the
simulation feature of TMB to conduct self- and cross-tests (\emph{sensu}
Deroba et al. 2015). For each model, we generated 100 sets of new data
and random effects, keeping the fixed effect parameters constant at
values estimated in original fits. We then re-fit all models to datasets
simulated under each as an operating model. We calculated the relative
error in parameters constraining random effects (Table
\ref{tab:model-descriptions}) and quantities of interest, such as
spawning stock biomass (SSB), \(F\),
\(\frac{\text{B}}{\text{B}_{40\%}}\), \(\frac{F}{F_{40\%}}\), and \(R\).
We calculated relative error as \(\frac{\hat{\theta_i}}{\theta_i}-1\),
where \(\theta_i\) is the true value for simulated dataset \(i\) and
\(\hat{\theta}_i\) is the value estimated from fitting the model to the
simulated data. To estimate bias of each estimation model for a given
operating model, we calculated 95\% confidence intervals of the median
relative error using the binomial distribution (Thompson 1936). To
summarize the bias across simulations and years for each model, we
calculated the median quantity across years and took the mean of the
medians across simulations. Finally, for each operating model we
calculated the proportion of simulations in which each estimation model
converged and had the lowest AIC.

\hypertarget{results}{%
\section{Results}\label{results}}

\hypertarget{original-datasets}{%
\subsection{Original datasets}\label{original-datasets}}

The 2D AR1 covariance structure for random effects on \emph{NAA},
\emph{M}, and selectivity performed well across stocks and processes,
evidenced by lower AIC than the IID random effects models (Fig.
\ref{fig:daic}). This AIC difference was larger for selectivity than
\emph{NAA} or \emph{M}, but the differences for \emph{NAA} and \emph{M}
were also non-trivial, ranging from 11.1--53.2. Incorporating random
effects on \emph{NAA}, \emph{M} or selectivity did not have consistently
positive or negative effects on key assessment model output (SSB,
\emph{F}, or recruitment); these effects differed by stock and process
(Figs. S1--S10). Models with random effects did have consistently wider
confidence intervals in SSB, \emph{F}, and recruitment (Figs. S1--S10).

\hypertarget{numbers-at-age-naa-1}{%
\subsubsection{\texorpdfstring{Numbers-at-age
(\emph{NAA})}{Numbers-at-age (NAA)}}\label{numbers-at-age-naa-1}}

Treating all ages as random effects was strongly supported by AIC,
compared to treating only recruitment deviations as random effects (Fig.
\ref{fig:daic}). Including autoregressive \emph{NAA} random effects was
also generally supported by AIC, either the AR1 when only recruitment
deviations were random effects or the 2D AR1 when all ages were random
effects. The estimated AR and IID random effects were similar, with the
AR random effects slightly smoothed compared to the IID random effects
(e.g.~for Icelandic herring in Fig. \ref{fig:devs-ICEherring-naa}).

\hypertarget{natural-mortality-m-1}{%
\subsubsection{\texorpdfstring{Natural mortality
(\emph{M})}{Natural mortality (M)}}\label{natural-mortality-m-1}}

As for the \emph{NAA} models, including random effect deviations on
\emph{M} was supported by AIC (Fig. \ref{fig:daic}). Although the 2D AR1
\emph{M} model did not converge for North Sea cod, it had the lowest AIC
for butterfish and SNEMA yellowtail flounder. In contrast to the
\emph{NAA} models, the patterns in estimated 2D AR1 and IID \emph{M}
random effects differed noticeably (e.g.~for butterfish in Fig.
\ref{fig:devs-butterfish-m}). The butterfish IID \emph{M} model
estimated elevated \emph{M} for age 5+ fish early and late in the
time-series, with lower \emph{M} in intervening years. The butterfish 2D
AR1 \emph{M} model estimated a similar, but much exaggerated, pattern
for age 5+ fish and reduced \emph{M} for ages 1 and 4. With \(\mu_M\)
held constant, negative \emph{M} deviations in some ages and years were
required to offset positive \emph{M} deviations in others.

\hypertarget{selectivity-s-1}{%
\subsubsection{\texorpdfstring{Selectivity
(\emph{s})}{Selectivity (s)}}\label{selectivity-s-1}}

For Georges Bank haddock, including 2D AR1 random effect deviations on
selectivity was strongly supported by AIC (Fig. \ref{fig:daic}). The 2D
AR1 \emph{s} model estimated similar patterns in selectivity compared to
the IID \emph{s} model, except with smoother variations by age and year
(Fig. \ref{fig:devs-GBhaddock-sel}). Compared to the model with constant
selectivity, the IID and 2D AR1 models estimated lower \emph{s} for age
3 in recent years and higher \emph{s} for age 3 before 1990. They also
estimated higher \emph{s} for age 2 before 1990, especially from 1973 to
1976.

\hypertarget{environmental-covariate-effect-on-recruitment-ecov}{%
\subsubsection{\texorpdfstring{Environmental covariate effect on
recruitment
(\emph{Ecov})}{Environmental covariate effect on recruitment (Ecov)}}\label{environmental-covariate-effect-on-recruitment-ecov}}

As in previous analyses (Miller et al. 2016; Xu et al. 2018), including
an effect of the CPI on recruitment for SNEMA yellowtail flounder was
clearly supported by AIC (\(\Delta \text{AIC}\) of 20.3-33.0 between m1
and m2-m4, Fig. \ref{fig:daic}). The AR1-linear model (m4) had the
lowest AIC---fitting the CPI using an AR1 model was preferred over the
random walk (\(\Delta \text{AIC}\) of 12.7 between m2 and m4), and the
quadratic term, \(\beta_2\), was deemed unnecessary
(\(\Delta \text{AIC}\) of 1.3 between m5 and m4). As expected, the AR1
process model estimated higher uncertainty in years with higher
observation error (e.g.~1982--1992) and missing observations (2017, Fig.
\ref{fig:devs-SNEMAYT-ecov}A). The CPI negatively influenced
recruitment, i.e. \(\hat{R}\) was higher following years with lower CPI
(lower fall bottom temperature, Fig. \ref{fig:devs-SNEMAYT-ecov}C).
Including the CPI-recruitment link changed the estimates of \(R_y\) by
up to 30\%, but in most years the relative difference in \(R_y\) was
less than 10\% (Figs. \ref{fig:devs-SNEMAYT-ecov}B--C and S3). The
CPI-recruitment link model had far less influence on estimates of SSB or
\emph{F}, less than 4\% difference in most years, especially compared to
models with random effects on all NAA or \emph{M} (Figs. S1--S3).

\hypertarget{simulation-tests-1}}\), \(\frac{F}{F_{40\%}}\), and \(R\)
was generally small and not significant based on confidence intervals
(Figs.
\ref{fig:rel-error-ICEherring-naa}--\ref{fig:rel-error-GBhaddock-sel}
and S11--S16). Bias was also generally small when more complex models
were fitted to less complicated operating model simulations. In
contrast, the bias was often non-trivial when the estimation model was
less complex than the operating model, especially when models without
random effects were fit to data simulated from models with random
effects. Bias in SSB and \(F\) were always opposite, i.e.~when SSB was
biased high, \(F\) was biased low, and vice versa. Predicted catch was
never biased. Bias of the variance and correlation parameters
controlling random effects was generally negligible or negative, as
expected (Figs. \ref{fig:estpar-naa}--\ref{fig:estpar-ecov}). Restricted
maximum likelihood (REML) should be used if more accurate estimation of
these parameters is a priority.

In cross-tests, the percentage of simulations in which AIC selected the
correct model varied between 62--99\% by process, stock, and operating
model (Fig. \ref{fig:aic-cross}). Estimation models more complex than
the operating model were more likely to be chosen for \emph{NAA} models
than for \emph{M} or selectivity.

\hypertarget{numbers-at-age}}\), \(\frac{F}{F_{40\%}}\), and \(R\)
with very minimal bias in self-tests. An exception was the estimation of
\(F\) for Icelandic herring, particularly for the SCAA models (median
relative error with 95\% CI for m1: -0.060 (-0.073, -0.047), m2: -0.060
(-0.073, -0.047), m3: 0.015 (-0.004, 0.034), and m4: 0.049 (0.017,
0.080); Figs. \ref{fig:rel-error-ICEherring-naa} and S17--S19). The
convergence rate for most models and stocks was above 95\%, again with
the exception of the SCAA models for Icelandic herring (Fig. S20). In
cross-tests, SCAA models exhibited non-trivial bias when estimating data
simulated from models that treated numbers at all ages as random
effects. The degree of bias varied by quantity and stock between -25\%
and 25\% (Figs. \ref{fig:rel-error-ICEherring-naa} and S11--S14). The
more complex NAA models estimated all quantities without bias regardless
of operating model.

For four of the five stocks, \(\sigma^2_R\) was estimated without bias
in self-tests (Fig. \ref{fig:estpar-naa}). The exception was butterfish,
for which \(\sigma^2_R\) was negatively biased in m1, m2, and m4 (but
not m3). In contrast, both m3 and m4 estimated \(\sigma^2_a\) with
negative bias for all five stocks. There was no consistent pattern in
bias for the correlation parameters \(\rho_a\) and \(\rho_y\).

\hypertarget{natural-mortality}}\),
\(\frac{F}{F_{40\%}}\), and \(R\) were estimated without significant
bias in self-tests, although less complex models exhibited bias when fit
to data simulated from more complex models (Figs.
\ref{fig:rel-error-butterfish-m} and S15--S16).

As for the \emph{NAA} models, \(\sigma^2_R\) was estimated without bias
in the \emph{M} model self-tests (Fig. \ref{fig:estpar-m}).
\(\sigma^2_M\) was biased low, \(\varphi_y\) had no bias, and the
direction of bias in \(\varphi_a\) was inconsistent.

\hypertarget{selectivity}}\), \(\frac{F}{F_{40\%}}\), and \(R\)
with little to no bias across operating models (Fig.
\ref{fig:rel-error-GBhaddock-sel}). The model without \emph{s} random
effects, m1, showed substantial bias when fit to data simulated from m2
or m3. For Georges Bank haddock, the variance and correlation parameters
\(\sigma^2_a\), \(\sigma^2_s\), \(\phi_y\), and \(\phi_a\) were all
estimated with slight negative bias in self-tests of models with
\emph{s} random effects (Fig. \ref{fig:estpar-sel}).

\hypertarget{ecov-recruitment}{%
\subsubsection{Ecov-Recruitment}\label{ecov-recruitment}}

The random walk CPI models estimated \(\sigma^2_X\) with less bias than
the AR1 CPI models (Fig. \ref{fig:estpar-ecov}). \(\beta_0\) was biased
high in all models, although this was not significant for the model with
lowest AIC (m4, AR1-linear). All models estimated the parameters
\(\phi_X\), \(\alpha\), \(\beta_1\), and \(\beta_2\) without significant
bias.

\hypertarget{discussion}{%
\section{Discussion}\label{discussion}}

\hypertarget{overview}{%
\subsection{Overview}\label{overview}}

Our results suggest that the WHAM package can be a useful tool for stock
assessment when environmental effects on recruitment, or stochastic
changes in the numbers at age transitions, selectivity, or natural
mortality are of interest. The simulation tests showed negligible or no
bias in estimation of important assessment outputs (SSB, \emph{F}, stock
and harvest status) when the operating and estimation models matched.
The less complex models, without random effects or autoregressive
structure, exhibited some bias in cross-tests, while the more complex
models did not. In these cases, bias in SSB and \emph{F} were opposite
such that predicted catch was unbiased. The WHAM models with IID or 2D
AR1 random effect deviations performed well across stocks and processes,
which suggests that they warrant consideration in future stock-specific
studies.

\hypertarget{relationships-to-other-existing-assessment-model-frameworks}{%
\subsection{Relationships to other existing assessment model
frameworks}\label{relationships-to-other-existing-assessment-model-frameworks}}

WHAM assumes separability in \(F_{a,y}\), i.e. \(F_{a,y} = F_y s_a\),
and estimates annual full \(F_y\) as fixed effects and \(s_a\) as
constant or time-varying random effects with various autoregressive
assumptions possible. Most other assessment frameworks in the U.S,
e.g.~ASAP, Stock Synthesis (SS; Methot and Wetzel 2013), an Assessment
Model for Alaska (AMAK; Anon. 2015), and the Beaufort Assessment Model
(BAM; Williams and Shertzer 2015), make this same separability
assumption which is useful for specifying a fully-selected \(F\) in
projections to calculate reference points or generating catch advice.
Estimating selectivity in time blocks is common practice in these
frameworks. In contrast, SAM estimates \(F_{a,y}\) directly as a
multivariate random walk process (Nielsen and Berg 2014). The
autoregressive models for selectivity parameters under certain
configurations of WHAM should allow for similar \(F\) at age patterns as
in SAM. Stock Synthesis allows 2D AR1 random effects on \(s_{a,y}\)
instead of the parameters, and then the variance parameter is estimated
by an iterative tuning algorithm (Xu et al. 2019). It is unclear whether
estimating time-varying selectivity as random effects produces better
results than assuming time blocks, or if there are advantages to using
any of the three approaches for estimating time-varying selectivity used
by WHAM, SAM, or SS.

Likewise, WHAM and the other U.S.-based assessment models treat catch,
index, and composition observations differently than SAM. The U.S.-based
assessment models treat aggregate and composition observations
separately for fisheries and indices whereas SAM treats observations of
catch and indices at age, \(C_{a,y}\) and \(I_{a,y}\), as multivariate
log-normal. Separate observation models for aggregate and composition
observations is natural when sampling for total catch differs from that
for length and age composition.

Like SAM, WHAM can estimate interannual transitions in NAA as a random
walk processes, but WHAM can also be configured to treat these
deviations as stationary autoregressive processes. Alternatively (or
simultaneously, Stock et al. n.d.) WHAM can estimate deviations in
natural mortality as autoregressive processes like NCAM (Cadigan 2016).
Uniquely, WHAM can model multiple environmental covariate time series as
state-space processes and include their effects, possibly nonlinearly,
in various ways on recruitment or natural mortality. Although most
applications thusfar investigate effects of physical processes on
demographic parameters, indices of predation might also be considered
(Marshall et al. 2019).

\hypertarget{future-use-of-wham}{%
\subsection{Future use of WHAM}\label{future-use-of-wham}}

The productivity of several groundfish stocks in the U.S. Northeast has
declined in recent decades, and conducting assessments in WHAM with
time-varying processes via random effects or environment-productivity
links could account for these trends and potentially reduce
retrospective bias (Perretti et al. 2017; Tableau et al. 2018; Stock et
al. n.d.). WHAM can be configured to fit SCAA models very similar to
ASAP and therefore bridge between the two frameworks. Adding random
effects or environmental covariates to an assessment model is a large
structural change, and simulation self- and cross-tests such as we have
demonstrated here should be conducted through the research track process
(Lynch et al. 2018). If comparisons against status quo SCAA models prove
favorable, WHAM could transition to being used in operational
assessments. However, because the details of how to include time-varying
processes, as well as the effect on the assessment output, will vary by
stock, this evaluation may need to be conducted on a stock-by-stock
basis.

Random effects allow for changing productivity and, if they are
considered as an autoregressive process, can propagate the effect of
these changes on assessment output in short-term projections (Stock et
al. n.d.). However, the AR1 process demonstrated here trends to the mean
in projections and will not predict values beyond extremes in observed
time-series. This is an issue worth examining because many marine
ecosystems are changing to such extent that recent conditions are
time-series extremes, and conditions in the near future may continue to
expand the range of observations (e.g., sea surface temperatures over
the U.S. Northeast Shelf in the last decade; Chen et al. 2020). More
complex nonstationary time-series models such as autoregressive
integrated moving average (ARIMA) that can forecast beyond observed
values could be implemented. Nonlinear (e.g.~splines) or double linear
integration (Di Lorenzo and Ohman 2013) techniques could also be worth
pursuing, as well as time-delay embedding methods (Munch et al. 2017,
2018), which not only allow for nonstationary dynamics but also do not
require a specified functional form. The best approach to making
stochastic projections of productivity responses to environmental
conditions that are rapidly changing and at time-series extremes merits
further research.

In addition to the empirical (i.e.~random effects) approach, it is worth
considering the mechanistic approach (i.e.~explicitly modeled links to
environmental covariates) for a given stock whose recruitment or
\emph{M} is suspected to have shifted due to a clear, plausible
hypothesized external influence. Several factors may result in a higher
likelihood that the mechanistic approach is useful in an assessment: a
history of overfishing (Free et al. 2019; but high \emph{F} can also
swamp the signal of an environmental influence, Haltuch and Punt 2011),
more rapid environmental change, stocks at the edge of species' range,
opportunistic (short-lived) species (\emph{sensu} Winemiller and Rose
1992), lower trophic level species, longer time series, periodic signals
in which more than once cycle has been recorded (e.g.~Pacific Decadal
Oscillation and sardine), and stronger signals (wider ranges of observed
stock status and environmental conditions) (Haltuch and Punt 2011;
Haltuch et al. 2019; Marshall et al. 2019; Free et al. 2019). Although
developing mature mechanistic hypotheses will take significant effort
and collaboration with ecologists and oceanographers, these rules of
thumb can guide researchers in prioritizing which stocks to investigate
environment-productivity relationships. Our ability to forecast and
understand oceanographic and climate variables continues to improve and
increasingly will present opportunities for including well-founded
mechanistic environmental effects in stock assessments (Tommasi et al.
2017).

The perception of precision in model output such as projected population
biomass or catch advice is affected by whether productivity componenents
are treated as either constant or stochastic processes (Figs. S1--S10),
much like whether certain parameters are either fixed or estimated in
traditional assessment models. Modeling frameworks such as WHAM allow
the performance of these alternative models to be compared and when
allowing temporal variation in productivity components is justified, the
greater uncertainty in the assessment output is more realistic.
Moreover, model uncertainty could be included in our preception of
output uncertainty by fitting ensembles of WHAM models that represent
alternative states of nature (Möllmann et al. 2014; Anderson et al.
2017). Finally, the simulation capabilities of WHAM can also be useful
in situations where variation in productivity is hypothesized, but
information to estimate this variation is unavialable. WHAM can be
configured as an operating model to simulate plausible
environmentally-driven changes in recruitment or \emph{M} or plausible
stochastic variation in order to estimate the sensitivity of status quo
models.

\hypertarget{conclusion}{%
\subsection{Conclusion}\label{conclusion}}

A major present-day challenge in fisheries is to assess and manage
stocks in a changing environment. We foresee environment-linked stock
assessments becoming more feasible and realistic as fisheries and
oceanographic time-series lengthen and our ecological understanding
deepens. We have developed WHAM with this in mind. Finally, we note that
the development of TMB has been a critical advance for fisheries
assessment modeling frameworks such as WHAM, allowing us to rapidly fit
models that treat population and environmental processes as time-varying
random effects in a state-space framework.

\hypertarget{acknowledgements}{%
\subsection*{Acknowledgements}\label{acknowledgements}}
\addcontentsline{toc}{subsection}{Acknowledgements}

This research was performed while BCS held an NRC Research Associateship
award at the NEFSC under the Northeast Groundfish and Climate
Initiative. We thank Chris Melrose for updating the Cold Pool Index with
data through 2018, using code originally written by Jon Hare.

\pagebreak

\hypertarget{references}{%
\subsection*{References}\label{references}}
\addcontentsline{toc}{subsection}{References}

\hypertarget{refs}{}
\leavevmode\hypertarget{ref-aeberhard2018Review}{}%
Aeberhard, W.H., Mills Flemming, J., and Nielsen, A. 2018. Review of
State-Space Models for Fisheries Science. Annu. Rev. Stat. Appl.
\textbf{5}(1): 215--235.
doi:\href{https://doi.org/10.1146/annurev-statistics-031017-100427}{10.1146/annurev-statistics-031017-100427}.

\leavevmode\hypertarget{ref-agostini2008Climateocean}{}%
Agostini, V., Hendrix, A., Hollowed, A., Wilson, C., Pierce, S., and
Francis, R. 2008. Climate-ocean variability and Pacific hake: A
geostatistical modeling approach. Journal of Marine Systems \textbf{71}:
237--248.
doi:\href{https://doi.org/10.1016/j.jmarsys.2007.01.010}{10.1016/j.jmarsys.2007.01.010}.

\leavevmode\hypertarget{ref-aldrin2020Specification}{}%
Aldrin, M., Tvete, I., Aanes, S., and Subbey, S. 2020. The specification
of the data model part in the SAM model matters. Fisheries Research
\textbf{229}: 105585.
doi:\href{https://doi.org/10.1016/j.fishres.2020.105585}{10.1016/j.fishres.2020.105585}.

\leavevmode\hypertarget{ref-anderson2017Improving}{}%
Anderson, S.C., Cooper, A.B., Jensen, O.P., Minto, C., Thorson, J.T.,
Walsh, J.C., Afflerbach, J., Dickey-Collas, M., Kleisner, K.M., Longo,
C., Osio, G.C., Ovando, D., Mosqueira, I., Rosenberg, A.A., and Selig,
E.R. 2017. Improving estimates of population status and trend with
superensemble models. Fish and Fisheries \textbf{18}(4): 732--741.
doi:\href{https://doi.org/10.1111/faf.12200}{10.1111/faf.12200}.

\leavevmode\hypertarget{ref-anon2015AMAK}{}%
Anon. 2015. Assessment Model for Alaska Description of GUI and
Instructions. Available from
\url{https://github.com/afsc-assessments/AMAK/blob/master/docs/AMAK\%20Documentation.pdf}.

\leavevmode\hypertarget{ref-berg2016Accounting}{}%
Berg, C.W., and Nielsen, A. 2016. Accounting for correlated observations
in an age-based state-space stock assessment model. ICES J Mar Sci
\textbf{73}(7): 1788--1797.
doi:\href{https://doi.org/10.1093/icesjms/fsw046}{10.1093/icesjms/fsw046}.

\leavevmode\hypertarget{ref-brett1971Energetic}{}%
Brett, J.R. 1971. Energetic responses of salmon to temperature. A study
of some thermal relations in the physiology and freshwater ecology of
sockeye salmon (\emph{Oncorhynchus} \emph{Nerka}). Am Zool
\textbf{11}(1): 99--113.
doi:\href{https://doi.org/10.1093/icb/11.1.99}{10.1093/icb/11.1.99}.

\leavevmode\hypertarget{ref-brooks2016Retrospective}{}%
Brooks, E.N., and Legault, C.M. 2016. Retrospective forecasting
evaluating performance of stock projections for New England groundfish
stocks. Can. J. Fish. Aquat. Sci. \textbf{73}(6): 935--950.
doi:\href{https://doi.org/10.1139/cjfas-2015-0163}{10.1139/cjfas-2015-0163}.

\leavevmode\hypertarget{ref-cadigan2016Statespace}{}%
Cadigan, N.G. 2016. A state-space stock assessment model for northern
cod, including under-reported catches and variable natural mortality
rates. Canadian Journal of Fisheries and Aquatic Sciences
\textbf{73}(2): 296--308.
doi:\href{https://doi.org/10.1139/cjfas-2015-0047}{10.1139/cjfas-2015-0047}.

\leavevmode\hypertarget{ref-chen2020Long}{}%
Chen, Z., Kwon, Y.-O., Chen, K., Fratantoni, P., Gawarkiewicz, G., and
Joyce, T.M. 2020. Long-Term SST Variability on the Northwest Atlantic
Continental Shelf and Slope. Geophys. Res. Lett. \textbf{47}(1).
doi:\href{https://doi.org/10.1029/2019GL085455}{10.1029/2019GL085455}.

\leavevmode\hypertarget{ref-deoliveira2005Limits}{}%
De Oliveira, J., and Butterworth, D. 2005. Limits to the use of
environmental indices to reduce risk and/or increase yield in the South
African anchovy fishery. African Journal of Marine Science
\textbf{27}(1): 191--203.
doi:\href{https://doi.org/10.2989/18142320509504078}{10.2989/18142320509504078}.

\leavevmode\hypertarget{ref-deroba2015Simulation}{}%
Deroba, J.J., Butterworth, D.S., Methot, R.D., De Oliveira, J.a.A.,
Fernandez, C., Nielsen, A., Cadrin, S.X., Dickey-Collas, M., Legault,
C.M., Ianelli, J., Valero, J.L., Needle, C.L., O'Malley, J.M., Chang,
Y.-J., Thompson, G.G., Canales, C., Swain, D.P., Miller, D.C.M.,
Hintzen, N.T., Bertignac, M., Ibaibarriaga, L., Silva, A., Murta, A.,
Kell, L.T., de Moor, C.L., Parma, A.M., Dichmont, C.M., Restrepo, V.R.,
Ye, Y., Jardim, E., Spencer, P.D., Hanselman, D.H., Blaylock, J., Mood,
M., and Hulson, P.-J.F. 2015. Simulation testing the robustness of stock
assessment models to error: Some results from the ICES strategic
initiative on stock assessment methods. ICES J Mar Sci \textbf{72}(1):
19--30.
doi:\href{https://doi.org/10.1093/icesjms/fst237}{10.1093/icesjms/fst237}.

\leavevmode\hypertarget{ref-dilorenzo2013Doubleintegration}{}%
Di Lorenzo, E., and Ohman, M.D. 2013. A double-integration hypothesis to
explain ocean ecosystem response to climate forcing. Proceedings of the
National Academy of Sciences \textbf{110}(7): 2496--2499.
doi:\href{https://doi.org/10.1073/pnas.1218022110}{10.1073/pnas.1218022110}.

\leavevmode\hypertarget{ref-free2019Impacts}{}%
Free, C.M., Thorson, J.T., Pinsky, M.L., Oken, K.L., Wiedenmann, J., and
Jensen, O.P. 2019. Impacts of historical warming on marine fisheries
production. Science \textbf{363}(6430): 979--983.
doi:\href{https://doi.org/10.1126/science.aau1758}{10.1126/science.aau1758}.

\leavevmode\hypertarget{ref-fry1971Effect}{}%
Fry, F. 1971. The effect of environmental factors on the physiology of
fish. \emph{In} Fish Physiology. Elsevier. pp. 1--98.
doi:\href{https://doi.org/10.1016/S1546-5098(08)60146-6}{10.1016/S1546-5098(08)60146-6}.

\leavevmode\hypertarget{ref-garstang1900Impoverishment}{}%
Garstang, W. 1900. The Impoverishment of the Sea. A Critical Summary of
the Experimental and Statistical Evidence bearing upon the Alleged
Depletion of the Trawling Grounds. Journal of the Marine Biological
Association of the United Kingdom \textbf{6}(01): 1--69.
doi:\href{https://doi.org/10.1017/S0025315400072374}{10.1017/S0025315400072374}.

\leavevmode\hypertarget{ref-gudmundsson1994Time}{}%
Gudmundsson, G. 1994. Time series analysis of catch-at-age observations.
Applied Statistics \textbf{43}(1): 117--126.

\leavevmode\hypertarget{ref-haltuch2019Unraveling}{}%
Haltuch, M.A., Brooks, E., Brodziak, J., Devine, J., Johnson, K.,
Klibansky, N., Nash, R., Payne, M., Shertzer, K., Subbey, S., and Wells,
B. 2019. Unraveling the recruitment problem: A review of
environmentally-informed forecasting and management strategy evaluation.
Fisheries Research \textbf{217}: 198--216.
doi:\href{https://doi.org/10.1016/j.fishres.2018.12.016}{10.1016/j.fishres.2018.12.016}.

\leavevmode\hypertarget{ref-haltuch2011Promises}{}%
Haltuch, M.A., and Punt, A.E. 2011. The promises and pitfalls of
including decadal-scale climate forcing of recruitment in groundfish
stock assessment. Canadian Journal of Fisheries and Aquatic Sciences
\textbf{68}(5): 912--926.
doi:\href{https://doi.org/10.1139/f2011-030}{10.1139/f2011-030}.

\leavevmode\hypertarget{ref-hare2016Northeast}{}%
Hare, J.A., Borggaard, D.L., Friedland, K.D., Anderson, J., Burns, P.,
Chu, K., Clay, P.M., Collins, M.J., Cooper, P., Fratantoni, P.S.,
Johnson, M.R., Manderson, J.P., Milke, L., Miller, T.J., Orphanides,
C.D., and Saba, V.S. 2016. Northeast Regional Action Plan - NOAA
Fisheries Climate Science Strategy. NOAA Fisheries, Northeast Fisheries
Science Center, Woods Hole, MA. Available from
\url{https://repository.library.noaa.gov/view/noaa/13138} {[}accessed 17
September 2019{]}.

\leavevmode\hypertarget{ref-hill2018Assessment}{}%
Hill, K.T., Crone, P.R., and Zwolinski, J.P. 2018. Assessment of the
Pacific sardine resource in 2018 for U.S. Management in 2018-2019. US
Department of Commerce. Available from
\href{https://\%20doi.org/10.7289/V5/TM-\%20SWFSC-600}{https:// doi.org/10.7289/V5/TM- SWFSC-600}.

\leavevmode\hypertarget{ref-hjort1914Fluctuations}{}%
Hjort, J. 1914. Fluctuations in the great fisheries of Northern Europe
viewed in the light of biological research. Rapports et Procès-Verbaux
des Réunions du Conseil Permanent International Pour L'Exploration de la
Mer \textbf{20}: 1--228.

\leavevmode\hypertarget{ref-ices2017Report}{}%
ICES. 2017a. Report of the working group on assessment of demersal
stocks in the North Sea and Skagerrak (2017). ICES CM 2017/ACOM:21, 26
April-5 May 2017, ICES HQ. Available from
\url{https://www.ices.dk/sites/pub/Publication/\%20Reports/Expert/\%20Group/\%20Report/acom/2017/WGNSSK/01/\%20WGNSSK/\%20Report/\%202017.pdf}.

\leavevmode\hypertarget{ref-ices2017Reporta}{}%
ICES. 2017b. Report of the North Western Working Group (NWWG). ICES CM
2017/ACOM:08, 27 April-4 May 2017, Copenhagen, Denmark. Available from
\url{http://www.ices.dk/sites/pub/Publication/\%20Reports/Expert/\%20Group/\%20Report/acom/2017/NWWG/NWWG/\%202017/\%20Report.pdf}.

\leavevmode\hypertarget{ref-ices2020Workshop}{}%
ICES. 2020. Workshop on the review and future of state space stock
assessment models in ICES (WKRFSAM). ICES Scientific Reports
\textbf{2}(32): 23p. ICES.
doi:\href{https://doi.org/10.17895/ices.pub.6004}{10.17895/ices.pub.6004}.

\leavevmode\hypertarget{ref-iles1998Stock}{}%
Iles, T.C., and Beverton, R.J.H. 1998. Stock, recruitment and moderating
processes in flatfish. Journal of Sea Research \textbf{39}(1): 41--55.
doi:\href{https://doi.org/10.1016/S1385-1101(97)00022-1}{10.1016/S1385-1101(97)00022-1}.

\leavevmode\hypertarget{ref-kristensen2016TMB}{}%
Kristensen, K., Nielsen, A., Berg, C., Skaug, H., and Bell, B.M. 2016.
TMB: Automatic differentiation and Laplace approximation. Journal of
Statistical Software \textbf{70}: 1--21.
doi:\href{https://doi.org/10.18637/jss.v070.i05}{10.18637/jss.v070.i05}.

\leavevmode\hypertarget{ref-larkin1996Concepts}{}%
Larkin, P. 1996. Concepts and issues in marine ecosystem management.
Reviews in Fish Biology and Fisheries \textbf{6}: 139--164.
doi:\href{https://doi.org/10.1007/BF00182341}{10.1007/BF00182341}.

\leavevmode\hypertarget{ref-legault1998Flexible}{}%
Legault, C.M., and Restrepo, V.R. 1998. A Flexible Forward
Age-Structured Assessment Program.

\leavevmode\hypertarget{ref-link2002What}{}%
Link, J.S. 2002. What Does Ecosystem-Based Fisheries Management Mean?
Fisheries \textbf{27}(4): 5.

\leavevmode\hypertarget{ref-lorenzen1996Relationship}{}%
Lorenzen, K. 1996. The relationship between body weight and natural
mortality in juvenile and adult fish: A comparison of natural ecosystems
and aquaculture. Journal of Fish Biology \textbf{49}(4): 627--642.
doi:\href{https://doi.org/10.1111/j.1095-8649.1996.tb00060.x}{10.1111/j.1095-8649.1996.tb00060.x}.

\leavevmode\hypertarget{ref-lynch2018Implementing}{}%
Lynch, P.D., Methot, R.D., and Link, J.S. (\emph{Editors}). 2018.
Implementing a Next Generation Stock Assessment Enterprise. An Update to
the NOAA Fisheries Stock Assessment Improvement Plan. U.S. Dep. Commer.,
NOAA Tech. Memo. NMFS-F/ SPO-183. p. 127.
doi:\href{https://doi.org/10.7755/TMSPO.183}{10.7755/TMSPO.183}.

\leavevmode\hypertarget{ref-marshall2019Inclusion}{}%
Marshall, K.N., Koehn, L.E., Levin, P.S., Essington, T.E., and Jensen,
O.P. 2019. Inclusion of ecosystem information in US fish stock
assessments suggests progress toward ecosystem-based fisheries
management. ICES J Mar Sci \textbf{76}(1): 1--9. Oxford Academic.
doi:\href{https://doi.org/10.1093/icesjms/fsy152}{10.1093/icesjms/fsy152}.

\leavevmode\hypertarget{ref-mcclatchie2010Reassessment}{}%
McClatchie, S., Goericke, R., Auad, G., and Hill, K. 2010. Re-assessment
of the stockRecruit and temperatureRecruit relationships for Pacific
sardine (Sardinops sagax). Can. J. Fish. Aquat. Sci. \textbf{67}(11):
1782--1790. NRC Research Press.
doi:\href{https://doi.org/10.1139/F10-101}{10.1139/F10-101}.

\leavevmode\hypertarget{ref-mendelssohn1988Problems}{}%
Mendelssohn, R. 1988. Some problems in estimating population sizes from
catch-at-age data. Fishery Bulletin \textbf{86}(4): 617--630.

\leavevmode\hypertarget{ref-methot2011Adjusting}{}%
Methot, R.D., and Taylor, I.G. 2011. Adjusting for bias due to
variability of estimated recruitments in fishery assessment models. Can.
J. Fish. Aquat. Sci. \textbf{68}(10): 1744--1760. NRC Research Press.
doi:\href{https://doi.org/10.1139/f2011-092}{10.1139/f2011-092}.

\leavevmode\hypertarget{ref-methot2013Stock}{}%
Methot, R.D., and Wetzel, C.R. 2013. Stock synthesis: A biological and
statistical framework for fish stock assessment and fishery management.
Fisheries Research \textbf{142}: 86--99.
doi:\href{https://doi.org/10.1016/j.fishres.2012.10.012}{10.1016/j.fishres.2012.10.012}.

\leavevmode\hypertarget{ref-miller2016Statespace}{}%
Miller, T.J., Hare, J.A., and Alade, L.A. 2016. A state-space approach
to incorporating environmental effects on recruitment in an
age-structured assessment model with an application to southern New
England yellowtail flounder. Canadian Journal of Fisheries and Aquatic
Sciences \textbf{73}(8): 1261--1270.
doi:\href{https://doi.org/10.1139/cjfas-2015-0339}{10.1139/cjfas-2015-0339}.

\leavevmode\hypertarget{ref-miller2018Evaluating}{}%
Miller, T.J., and Hyun, S.-Y. 2018. Evaluating evidence for alternative
natural mortality and process error assumptions using a state-space,
age-structured assessment model. Canadian Journal of Fisheries and
Aquatic Sciences \textbf{75}(5): 691--703.
doi:\href{https://doi.org/10.1139/cjfas-2017-0035}{10.1139/cjfas-2017-0035}.

\leavevmode\hypertarget{ref-miller2015Technical}{}%
Miller, T.J., and Legault, C.M. 2015. Technical details for ASAP version
4. US Dept Commer, Northeast Fish Sci Cent. Available from
\url{doi:10.7289/V57W695G} {[}accessed 17 October 2019{]}.

\leavevmode\hypertarget{ref-miller2018Temporal}{}%
Miller, T.J., O'Brien, L., and Fratantoni, P.S. 2018. Temporal and
environmental variation in growth and maturity and effects on management
reference points of Georges Bank Atlantic cod. Can. J. Fish. Aquat.
Sci.: 1--13.
doi:\href{https://doi.org/10.1139/cjfas-2017-0124}{10.1139/cjfas-2017-0124}.

\leavevmode\hypertarget{ref-miller2020Woods}{}%
Miller, T.J., and Stock, B.C. 2020. The Woods Hole Assessment Model
(WHAM). Available from \url{https://timjmiller.github.io/wham/}.

\leavevmode\hypertarget{ref-mollmann2014Implementing}{}%
Möllmann, C., Lindegren, M., Blenckner, T., Bergström, L., Casini, M.,
Diekmann, R., Flinkman, J., Müller-Karulis, B., Neuenfeldt, S., Schmidt,
J.O., Tomczak, M., Voss, R., and Gårdmark, A. 2014. Implementing
ecosystem-based fisheries management: From single-species to integrated
ecosystem assessment and advice for Baltic Sea fish stocks. ICES J Mar
Sci \textbf{71}(5): 1187--1197. Oxford Academic.
doi:\href{https://doi.org/10.1093/icesjms/fst123}{10.1093/icesjms/fst123}.

\leavevmode\hypertarget{ref-munch2018Nonlinear}{}%
Munch, S.B., Giron-Nava, A., and Sugihara, G. 2018. Nonlinear dynamics
and noise in fisheries recruitment: A global meta-analysis. Fish and
Fisheries \textbf{19}: 964--973.
doi:\href{https://doi.org/10.1111/faf.12304}{10.1111/faf.12304}.

\leavevmode\hypertarget{ref-munch2017Circumventing}{}%
Munch, S.B., Poynor, V., and Arriaza, J.L. 2017. Circumventing
structural uncertainty: A Bayesian perspective on nonlinear forecasting
for ecology. Ecological Complexity \textbf{32}: 134--143.
doi:\href{https://doi.org/10.1016/j.ecocom.2016.08.006}{10.1016/j.ecocom.2016.08.006}.

\leavevmode\hypertarget{ref-myers1998When}{}%
Myers, R.A. 1998. When do environment-recruitment correlations work?
Reviews in Fish Biology and Fisheries \textbf{8}: 285--305.

\leavevmode\hypertarget{ref-nefsc2020Operational}{}%
NEFSC. 2020a. Operational assessment of 14 Northeast groundfish stocks,
updated through 2018. U.S. Dept. Commer., NOAA, NMFS, NEFSC, Woods Hole,
MA. Available from
\url{https://nefsc.noaa.gov/saw/2019-groundfish-docs/Prepublication-NE-Grndfsh-1-7-2020.pdf}
{[}accessed 5 August 2020{]}.

\leavevmode\hypertarget{ref-nefsc2020Butterfish}{}%
NEFSC. 2020b. Butterfish 2020 assessment update report. U.S. Dept.
Commer., NOAA, NMFS, NEFSC, Woods Hole, MA.

\leavevmode\hypertarget{ref-nielsen2014Estimation}{}%
Nielsen, A., and Berg, C.W. 2014. Estimation of time-varying selectivity
in stock assessments using state-space models. Fisheries Research
\textbf{158}: 96--101.
doi:\href{https://doi.org/10.1016/j.fishres.2014.01.014}{10.1016/j.fishres.2014.01.014}.

\leavevmode\hypertarget{ref-oleary2019Understanding}{}%
O'Leary, C.A., Miller, T.J., Thorson, J.T., and Nye, J.A. 2019.
Understanding historical summer flounder ( \emph{Paralichthys}
\emph{Dentatus} ) abundance patterns through the incorporation of
oceanography-dependent vital rates in Bayesian hierarchical models. Can.
J. Fish. Aquat. Sci. \textbf{76}(8): 1275--1294.
doi:\href{https://doi.org/10.1139/cjfas-2018-0092}{10.1139/cjfas-2018-0092}.

\leavevmode\hypertarget{ref-patrick2015Myths}{}%
Patrick, W.S., and Link, J.S. 2015. Myths that Continue to Impede
Progress in Ecosystem-Based Fisheries Management. Fisheries
\textbf{40}(4): 155--160.
doi:\href{https://doi.org/10.1080/03632415.2015.1024308}{10.1080/03632415.2015.1024308}.

\leavevmode\hypertarget{ref-perretti2017Regime}{}%
Perretti, C.T., Fogarty, M.J., Friedland, K.D., Hare, J.A., Lucey, S.M.,
McBride, R.S., Miller, T.J., Morse, R.E., O'Brien, L., Pereira, J.J.,
Smith, L.A., and Wuenschel, M.J. 2017. Regime shifts in fish recruitment
on the Northeast US Continental Shelf. Marine Ecology Progress Series
\textbf{574}: 1--11.
doi:\href{https://doi.org/10.3354/meps12183}{10.3354/meps12183}.

\leavevmode\hypertarget{ref-pershing2015Slow}{}%
Pershing, A.J., Alexander, M.A., Hernandez, C.M., Kerr, L.A., Bris,
A.L., Mills, K.E., Nye, J.A., Record, N.R., Scannell, H.A., Scott, J.D.,
Sherwood, G.D., and Thomas, A.C. 2015. Slow adaptation in the face of
rapid warming leads to collapse of the Gulf of Maine cod fishery.
Science \textbf{350}(6262): 809--812.
doi:\href{https://doi.org/10.1126/science.aac9819}{10.1126/science.aac9819}.

\leavevmode\hypertarget{ref-punt2014Fisheries}{}%
Punt, A.E., A'mar, T., Bond, N.A., Butterworth, D.S., de Moor, C.L., De
Oliveira, J.A.A., Haltuch, M.A., Hollowed, A.B., and Szuwalski, C. 2014.
Fisheries management under climate and environmental uncertainty:
Control rules and performance simulation. ICES J Mar Sci \textbf{71}(8):
2208--2220.
doi:\href{https://doi.org/10.1093/icesjms/fst057}{10.1093/icesjms/fst057}.

\leavevmode\hypertarget{ref-rcoreteam2020Language}{}%
R Core Team. 2020. R: A Language and Environment for Statistical
Computing. R Foundation for Statistical Computing, Vienna, Austria.
Available from \url{https://www.R-project.org}.

\leavevmode\hypertarget{ref-rose2015Northern}{}%
Rose, G.A., and Rowe, S. 2015. Northern cod comeback. Can. J. Fish.
Aquat. Sci. \textbf{72}(12): 1789--1798.
doi:\href{https://doi.org/10.1139/cjfas-2015-0346}{10.1139/cjfas-2015-0346}.

\leavevmode\hypertarget{ref-shelton2006Fishing}{}%
Shelton, P.A., Sinclair, A.F., Chouinard, G.A., Mohn, R., and Duplisea,
D.E. 2006. Fishing under low productivity conditions is further delaying
recovery of Northwest Atlantic cod (Gadus morhua). Can. J. Fish. Aquat.
Sci. \textbf{63}(2): 235--238.
doi:\href{https://doi.org/10.1139/f05-253}{10.1139/f05-253}.

\leavevmode\hypertarget{ref-shotwell2014Biophysical}{}%
Shotwell, S.K., Hanselman, D.H., and Belkin, I.M. 2014. Toward
biophysical synergy: Investigating advection along the Polar Front to
identify factors influencing Alaska sablefish recruitment. Deep Sea
Research Part II: Topical Studies in Oceanography \textbf{107}: 40--53.
doi:\href{https://doi.org/10.1016/j.dsr2.2012.08.024}{10.1016/j.dsr2.2012.08.024}.

\leavevmode\hypertarget{ref-stockthisissueImplementing}{}%
Stock, B.C., Xu, H., Miller, T.J., Thorson, J.T., and Nye, J.A. (n.d.).
Implementing a 2-dimensional smoother on either survival or natural
mortality improves a state-space assessment model for Southern New
England-Mid Atlantic yellowtail flounder.

\leavevmode\hypertarget{ref-stock2011Use}{}%
Stock, C.A., Alexander, M.A., Bond, N.A., Brander, K.M., Cheung, W.W.,
Curchitser, E.N., Delworth, T.L., Dunne, J.P., Griffies, S.M., Haltuch,
M.A., Hare, J.A., Hollowed, A.B., Lehodey, P., Levin, S.A., Link, J.S.,
Rose, K.A., Rykaczewski, R.R., Sarmiento, J.L., Stouffer, R.J., Schwing,
F.B., Vecchi, G.A., and Werner, F.E. 2011. On the use of IPCC-class
models to assess the impact of climate on Living Marine Resources.
Progress in Oceanography \textbf{88}(1-4): 1--27.
doi:\href{https://doi.org/10.1016/j.pocean.2010.09.001}{10.1016/j.pocean.2010.09.001}.

\leavevmode\hypertarget{ref-sullivan1992Kalman}{}%
Sullivan, P.J. 1992. A Kalman filter approach to catch-at-length
analysis. Biometrics \textbf{48}(1): 237--257.

\leavevmode\hypertarget{ref-tableau2018Decadal}{}%
Tableau, A., Collie, J.S., Bell, R.J., and Minto, C. 2018. Decadal
changes in the productivity of New England fish populations. Can. J.
Fish. Aquat. Sci. \textbf{76}(9): 1528--1540.
doi:\href{https://doi.org/10.1139/cjfas-2018-0255}{10.1139/cjfas-2018-0255}.

\leavevmode\hypertarget{ref-thompson1936Confidence}{}%
Thompson, W.R. 1936. On Confidence Ranges for the Median and Other
Expectation Distributions for Populations of Unknown Distribution Form.
Ann. Math. Statist. \textbf{7}(3): 122--128.
doi:\href{https://doi.org/10.1214/aoms/1177732502}{10.1214/aoms/1177732502}.

\leavevmode\hypertarget{ref-thorson2019Perspective}{}%
Thorson, J.T. 2019. Perspective: Let's simplify stock assessment by
replacing tuning algorithms with statistics. Fisheries Research
\textbf{217}: 133--139.
doi:\href{https://doi.org/10.1016/j.fishres.2018.02.005}{10.1016/j.fishres.2018.02.005}.

\leavevmode\hypertarget{ref-tommasi2017Managing}{}%
Tommasi, D., Stock, C.A., Hobday, A.J., Methot, R., Kaplan, I.C.,
Eveson, J.P., Holsman, K., Miller, T.J., Gaichas, S., Gehlen, M.,
Pershing, A., Vecchi, G.A., Msadek, R., Delworth, T., Eakin, C.M.,
Haltuch, M.A., Séférian, R., Spillman, C.M., Hartog, J.R., Siedlecki,
S., Samhouri, J.F., Muhling, B., Asch, R.G., Pinsky, M.L., Saba, V.S.,
Kapnick, S.B., Gaitan, C.F., Rykaczewski, R.R., Alexander, M.A., Xue,
Y., Pegion, K.V., Lynch, P., Payne, M.R., Kristiansen, T., Lehodey, P.,
and Werner, F.E. 2017. Managing living marine resources in a dynamic
environment: The role of seasonal to decadal climate forecasts. Progress
in Oceanography \textbf{152}: 15--49.
doi:\href{https://doi.org/10.1016/j.pocean.2016.12.011}{10.1016/j.pocean.2016.12.011}.

\leavevmode\hypertarget{ref-walters1988Research}{}%
Walters, C.J., and Collie, J.S. 1988. Is Research on Environmental
Factors Useful to Fisheries Management? Can. J. Fish. Aquat. Sci.
\textbf{45}(10): 1848--1854.
doi:\href{https://doi.org/10.1139/f88-217}{10.1139/f88-217}.

\leavevmode\hypertarget{ref-williams2015BAM}{}%
Williams, E.H., and Shertzer, K.W. 2015. Technical documentation of the
Beaufort Assessment Model (BAM). NOAA Technical Memorandum
NMFS-SEFSC-671. 43pp.

\leavevmode\hypertarget{ref-winemiller1992Patterns}{}%
Winemiller, K.O., and Rose, K.A. 1992. Patterns of Life-History
Diversification in North American Fishes: Implications for Population
Regulation. Canadian Journal of Fisheries and Aquatic Sciences
\textbf{49}(10): 2196--2218.
doi:\href{https://doi.org/10.1139/f92-242}{10.1139/f92-242}.

\leavevmode\hypertarget{ref-xu2018Evaluating}{}%
Xu, H., Miller, T.J., Hameed, S., Alade, L.A., and Nye, J.A. 2018.
Evaluating the utility of the Gulf Stream Index for predicting
recruitment of Southern New England-Mid Atlantic yellowtail flounder.
Fisheries Oceanography \textbf{27}(1): 85--95.
doi:\href{https://doi.org/10.1111/fog.12236}{10.1111/fog.12236}.

\leavevmode\hypertarget{ref-xu2020Comparing}{}%
Xu, H., Thorson, J.T., and Methot, R.D. 2020. Comparing the performance
of three data weighting methods when allowing for time-varying
selectivity. Can. J. Fish. Aquat. Sci. \textbf{77}(2): 247--263.
doi:\href{https://doi.org/10.1139/cjfas-2019-0107}{10.1139/cjfas-2019-0107}.

\leavevmode\hypertarget{ref-xu2019New}{}%
Xu, H., Thorson, J.T., Methot, R.D., and Taylor, I.G. 2019. A new
semi-parametric method for autocorrelated age- and time-varying
selectivity in age-structured assessment models. Can. J. Fish. Aquat.
Sci. \textbf{76}(2): 268--285.
doi:\href{https://doi.org/10.1139/cjfas-2017-0446}{10.1139/cjfas-2017-0446}.

\leavevmode\hypertarget{ref-zwolinski2012Cold}{}%
Zwolinski, J.P., and Demer, D.A. 2012. A cold oceanographic regime with
high exploitation rates in the Northeast Pacific forecasts a collapse of
the sardine stock. Proceedings of the National Academy of Sciences
\textbf{109}(11): 4175--4180.
doi:\href{https://doi.org/10.1073/pnas.1113806109}{10.1073/pnas.1113806109}.

\pagebreak

\begin{landscape}
\begin{table}

\caption{\label{tab:model-descriptions}Model descriptions and estimated parameters. Parameter descriptions and equations are given in text. Note that the base model in the $M$ module is NAA m1, and the base model in the Selectivity and Ecov-Recruitment modules is NAA m3. Ecov m1 fits the Cold Pool Index data and estimates $\sigma_x$ in order to allow comparison to m2-m5 using AIC (same data needed in likelihood).}
\centering
\begin{tabular}[t]{lll}
\toprule
Model & Description & Estimated parameters\\
\midrule
\addlinespace[0.3em]
\multicolumn{3}{l}{\textbf{Numbers-at-age (NAA)}}\\
\hspace{1em}m1: SCAA (IID) & Recruitment deviations are IID random effects & $\sigma_R$\\
\hspace{1em}m2: SCAA (AR1) & Recruitment deviations are autocorrelated (AR1) random effects & $\sigma_R$, $\rho_y$\\
\hspace{1em}m3: NAA (IID) & All NAA deviations are IID random effects & $\sigma_R$, $\sigma_a$\\
\hspace{1em}m4: NAA (2D AR1) & All NAA deviations are random effects with correlation by year and age (2D AR1) & $\sigma_R$, $\sigma_a$, $\rho_y$, $\rho_a$\\
\addlinespace[0.3em]
\multicolumn{3}{l}{\textbf{Natural mortality ($M$)}}\\
\hspace{1em}m1: none & No random effects on $M$ & $\sigma_R$\\
\hspace{1em}m2: IID & $M$ deviations are IID random effects & $\sigma_R$, $\sigma_M$\\
\hspace{1em}m3: 2D AR1 & $M$ deviations are random effects with correlation by year and age (2D AR1) & $\sigma_R$, $\sigma_M$, $\varphi_y$, $\varphi_a$\\
\addlinespace[0.3em]
\multicolumn{3}{l}{\textbf{Selectivity (Sel)}}\\
\hspace{1em}m1: none & No random effects on selectivity & $\sigma_R$, $\sigma_a$\\
\hspace{1em}m2: IID & Selectivity deviations are IID random effects & $\sigma_R$, $\sigma_a$, $\sigma_{Sel}$\\
\hspace{1em}m3: 2D AR1 & Selectivity deviations are random effects with correlation by year and age (2D AR1) & $\sigma_R$, $\sigma_a$, $\sigma_{Sel}$, $\phi_y$, $\phi_a$\\
\addlinespace[0.3em]
\multicolumn{3}{l}{\textbf{Ecov-Recruitment (Ecov)}}\\
\hspace{1em}m1: RW-none & Ecov: random walk (RW), effect on $\beta$: none & $\sigma_R$, $\sigma_a$, $\sigma_x$\\
\hspace{1em}m2: RW-linear & Ecov: random walk (RW), effect on $\beta$: linear & $\sigma_R$, $\sigma_a$, $\sigma_x$, $\beta_1$\\
\hspace{1em}m3: RW-poly & Ecov: random walk (RW), effect on $\beta$: 2nd order polynomial (poly) & $\sigma_R$, $\sigma_a$, $\sigma_x$, $\beta_1$, $\beta_2$\\
\hspace{1em}m4: AR1-linear & Ecov: autocorrelated (AR1), effect on $\beta$: linear & $\sigma_R$, $\sigma_a$, $\sigma_x$, $\phi_x$, $\beta_1$\\
\hspace{1em}m5: AR1-poly & Ecov: autocorrelated (AR1), effect on $\beta$: 2nd order polynomial (poly) & $\sigma_R$, $\sigma_a$, $\sigma_x$, $\phi_x$, $\beta_1$, $\beta_2$\\
\bottomrule
\end{tabular}
\end{table}
\end{landscape}

\pagebreak

\begin{landscape}
\begin{table}

\caption{\label{tab:stock-list}Stocks used in simulation tests.}
\centering
\begin{tabular}[t]{lllllrrlrrrl}
\toprule
\multicolumn{1}{c}{ } & \multicolumn{4}{c}{Processes tested} & \multicolumn{2}{c}{Model dim} & \multicolumn{2}{c}{Biol. par.} & \multicolumn{2}{c}{Stock status} \\
\cmidrule(l{3pt}r{3pt}){2-5} \cmidrule(l{3pt}r{3pt}){6-7} \cmidrule(l{3pt}r{3pt}){8-9} \cmidrule(l{3pt}r{3pt}){10-11}
Stock & NAA & M & Sel & Ecov & \# Ages & \# Years & $M$ & $\sigma_R$ & $\frac{B}{B_{40}}$ & $\frac{F}{F_{40}}$ & Source\\
\midrule
SNEMA yellowtail flounder & x & x &  & x & 6 & 49 & 0.2-0.4 & 1.67 & 0.01 & 0.44 & NEFSC (2020a)\\
Butterfish & x & x &  &  & 5 & 31 & 1.3 & 0.23 & 2.57 & 0.03 & NEFSC (2020b)\\
North Sea cod & x & x &  &  & 6 & 54 & 0.2-1.2 & 0.87 & 0.14 & 2.00 & ICES (2017a)\\
Icelandic herring & x &  &  &  & 11 & 30 & 0.1 & 0.55 & 0.40 & 1.81 & ICES (2017b)\\
Georges Bank haddock & x &  & x &  & 9 & 86 & 0.2 & 1.65 & 5.16 & 0.12 & NEFSC (2020a)\\
\bottomrule
\end{tabular}
\end{table}
\end{landscape}

\pagebreak

\begin{landscape}
\begin{figure}

{\centering \includegraphics[width=8in]{/home/bstock/Documents/ms/wham-sim/plots/into_paper/daic} 

}

\caption{AIC differences by model and stock when fit to original datasets. Stock abbreviations: SNEMA yellowtail flounder (SNEMAYT), North Sea cod (NScod), Icelandic herring (ICEherring), and Georges Bank haddock (GBhaddock).}\label{fig:daic}
\end{figure}
\end{landscape}

\pagebreak

\begin{figure}

{\centering \includegraphics[width=6.5in]{/home/bstock/Documents/ms/wham-sim/plots/into_paper/8_REdevs_ICEherring_NAA} 

}

\caption{Abundanc-at-age deviations estimated for Icelandic herring using four models of numbers-at-age (NAA) random effects. m1 = only recruitment deviations are random effects (most similar to traditional statistical catch-at-age, SCAA), and deviations are independent and identically distributed (IID). m2 = as m1, but with autocorrelated recruitment deviations ($\text{AR1}_y$). m3 = all NAA deviations are IID random effects. m4 = as m3, but deviations are correlated by age and year (2D AR1).}\label{fig:devs-ICEherring-naa}
\end{figure}

\pagebreak

\begin{figure}

{\centering \includegraphics[width=6in]{/home/bstock/Documents/ms/wham-sim/plots/into_paper/8_REdevs_butterfish_M} 

}

\caption{Natural mortality (\textit{M}) estimated for butterfish using three random effects models. m1 = no random effects on \textit{M}. m2 = \textit{M} deviations are independent and identically distributed (IID). m3 = \textit{M} deviations are correlated by age and year (2D AR1).}\label{fig:devs-butterfish-m}
\end{figure}

\pagebreak

\begin{figure}

{\centering \includegraphics[width=6.5in]{/home/bstock/Documents/ms/wham-sim/plots/into_paper/8_REdevs_GBhaddock_sel} 

}

\caption{Selectivity estimated for Georges Bank haddock using three random effects models. m1 = no random effects (constant logistic selectivity). m2 = selectivity deviations are independent and identically distributed (IID). m3 = selectivity deviations are correlated by parameter and year (2D AR1).}\label{fig:devs-GBhaddock-sel}
\end{figure}

\pagebreak

\begin{figure}

{\centering \includegraphics[width=6.5in]{/home/bstock/Documents/ms/wham-sim/plots/into_paper/8_CPI_Recruit_SNEMAYT_Ecov2} 

}

\caption{Beverton-Holt stock-recruit relationships fit for Southern New England-Mid Atlantic yellowtail flounder, with and without effects of the Cold Pool Index (CPI). A) CPI estimated from the model with lowest AIC (m4, AR1-linear). Points are observations with 95\% CI, and the line with shading is the model-estimated CPI with 95\% CI. Note the increased uncertainty surrounding the CPI estimate in 2017 (no observation). B) Estimates of spawning stock biomass (SSB), recruitment, and the stock-recruit function from the model without a CPI effect, m1. C) Estimates of SSB and recruitment from m4, with an effect of the CPI on $\beta$. Lines depict the expected stock-recruit relationship in each year $t$, given the CPI in year $t-1$ (color).}\label{fig:devs-SNEMAYT-ecov}
\end{figure}

\pagebreak

\begin{figure}

{\centering \includegraphics[width=6.5in]{/home/bstock/Documents/ms/wham-sim/plots/into_paper/0_ICEherring_NAA_medianCI_OEPE} 

}

\caption{Relative error of key quantities estimated for Icelandic herring using four models of numbers-at-age (NAA) random effects. m1 = only recruitment deviations are random effects (most similar to traditional statistical catch-at-age, SCAA), and deviations are independent and identically distributed (IID). m2 = as m1, but with autocorrelated recruitment deviations ($\text{AR1}_y$). m3 = all NAA deviations are IID random effects. m4 = as m3, but deviations are correlated by age and year (2D AR1).}\label{fig:rel-error-ICEherring-naa}
\end{figure}

\pagebreak

\begin{figure}

{\centering \includegraphics[width=5in]{/home/bstock/Documents/ms/wham-sim/plots/into_paper/0_butterfish_M_medianCI_OEPE} 

}

\caption{Relative error of key quantities estimated for butterfish using three models of natural mortality (\textit{M}) random effects. m1 = no random effects on \textit{M}. m2 = \textit{M} deviations are independent and identically distributed (IID). m3 = \textit{M} deviations are correlated by age and year (2D AR1).}\label{fig:rel-error-butterfish-m}
\end{figure}

\pagebreak

\begin{figure}

{\centering \includegraphics[width=5in]{/home/bstock/Documents/ms/wham-sim/plots/into_paper/0_GBhaddock_sel_medianCI_OEPE} 

}

\caption{Relative error of key quantities estimated for Georges Bank haddock using three models of selectivity random effects. m1 = no random effects (constant logistic selectivity). m2 = selectivity deviations are independent and identically distributed (IID). m3 = selectivity deviations are correlated by parameter and year (2D AR1).}\label{fig:rel-error-GBhaddock-sel}
\end{figure}

\pagebreak

\begin{figure}

{\centering \includegraphics[width=6.5in]{/home/bstock/Documents/ms/wham-sim/plots/into_paper/estpar_NAA_OEPE} 

}

\caption{Relative error of parameters constraining numbers-at-age (NAA) random effects. Four models were used to simulate 100 datasets keeping fixed effect parameters constant, and then re-fit to each simulated dataset. m1 = only recruitment deviations are random effects (most similar to traditional statistical catch-at-age, SCAA), and deviations are independent and identically distributed (IID). m2 = as m1, but with autocorrelated recruitment deviations ($\text{AR1}_y$). m3 = all NAA deviations are IID random effects. m4 = as m3, but deviations are correlated by age and year (2D AR1). Relative error was calculated as $\frac{\hat{\theta_i}}{\theta} - 1$, where $\hat{\theta_i}$ was the estimate in simulation $i$ for parameter $\theta$, and $\theta$ was the true value (estimate from original dataset). Red points and lines show median relative error with 95\% CI.}\label{fig:estpar-naa}
\end{figure}

\pagebreak

\begin{figure}

{\centering \includegraphics[width=6in]{/home/bstock/Documents/ms/wham-sim/plots/into_paper/estpar_M_4par_OEPE} 

}

\caption{Relative error of parameters constraining natural mortality (\textit{M}) random effects. Three models were used to simulate 100 datasets keeping fixed effect parameters constant, and then re-fit to each simulated dataset. m1 = no random effects on \textit{M}. m2 = \textit{M} deviations were independent and identically distributed (IID). m3 = \textit{M} deviations were correlated by age and year (2D AR1). Relative error was calculated as $\frac{\hat{\theta_i}}{\theta} - 1$, where $\hat{\theta_i}$ was the estimate in simulation $i$ for parameter $\theta$, and $\theta$ was the true value (estimate from original dataset). Red points and lines show median relative error with 95\% CI. Stock abbreviations: SNEMA yellowtail flounder (SNEMAYT) and North Sea cod (NScod, m3 did not converge).}\label{fig:estpar-m}
\end{figure}

\pagebreak

\begin{figure}

{\centering \includegraphics[width=6.5in]{/home/bstock/Documents/ms/wham-sim/plots/into_paper/estpar_sel_OEPE} 

}

\caption{Relative error of parameters constraining selectivity random effects for Georges Bank haddock (GBhaddock). Three models were used to simulate 100 datasets keeping fixed effect parameters constant, and then re-fit to each simulated dataset. m1 = no random effects (constant selectivity). m2 = selectivity deviations were independent and identically distributed (IID). m3 = selectivity deviations were correlated by parameter and year (2D AR1). Relative error was calculated as $\frac{\hat{\theta_i}}{\theta} - 1$, where $\hat{\theta_i}$ was the estimate in simulation $i$ for parameter $\theta$, and $\theta$ was the true value (estimate from original dataset). Red points and lines show median relative error with 95\% CI.}\label{fig:estpar-sel}
\end{figure}

\pagebreak

\begin{landscape}
\begin{figure}

{\centering \includegraphics[width=9in]{/home/bstock/Documents/ms/wham-sim/plots/into_paper/estpar_Ecov2_OEPE} 

}

\caption{Relative error of parameters constraining variation in recruitment for Southern New England-Mid Atlantic yellowtail flounder (SNEMAYT). Five models were used to simulate 100 datasets keeping fixed effect parameters constant, and then re-fit to each simulated dataset. All models estimated recruitment using the Beverton-Holt function and included CPI effects on $\beta$: $\hat{R}_{t+1} = \frac{\alpha S_{t}}{1 + e^{\beta_0 + \beta_1 x_{t} + \beta_2 x^2_{t}} S_t}$. m1 = Cold Pool Index (CPI) modeled as a random walk (RW) with no effect on recruitment ($\beta_1 = \beta_2 = 0$). m2 = CPI as RW, linear effect on $\beta$. m3 = CPI as RW, 2nd order polynomial effect on $\beta$. m4 = CPI as AR1, linear effect. m5 = CPI as AR1, polynomial effect. Relative error was calculated as $\frac{\hat{\theta_i}}{\theta} - 1$, where $\hat{\theta_i}$ was the estimate in simulation $i$ for parameter $\theta$, and $\theta$ was the true value (estimate from original dataset). Red points and lines show median relative error with 95\% CI.}\label{fig:estpar-ecov}
\end{figure}
\end{landscape}

\pagebreak

\begin{figure}

{\centering \includegraphics[width=3.5in]{/home/bstock/Documents/ms/wham-sim/plots/into_paper/aic_cross_multipanel} 

}

\caption{Proportion of simulations in which each model had the lowest AIC. A) Numbers-at-age (NAA), aggregated across all five stocks. B) Natural mortality (\textit{M}), aggregated over two stocks (SNEMAYT and butterfish). C) Selectivity (GBhaddock). Not all estimation models converged for each simulation, even when the operating model matched.}\label{fig:aic-cross}
\end{figure}

\end{document}
